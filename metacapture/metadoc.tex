\DocumentMetadata{pdfstandard=a-2u,pdfversion=1.7,lang=en,debug={xmp-export}}
\def\pkgversion{0.9}
%%%%%%%%%%%%%%%%%%%%%%%%%%%%%%%%%%%%%%%%%%%%%%%%%%%%%
% It should work with many choices of documentclass.
\documentclass{article}
%\documentclass[twocolumn]{article}
%\documentclass[journal=tches,submission]{iacrtrans}
%\documentclass[twocolumn]{acmart}
%\documentclass{acmart}
%\documentclass[format=acmsmall]{acmart}
%\documentclass[onecolumn,conference,compsoc]{IEEEtran}
% maketitle is a style mismatch for amsart.
%\documentclass{amsart}
% This puts footnotes as endnotes. and maketitle does not work.
%\documentclass[aps,prl,preprint,nofootinbib]{revtex4-2}
%\documentclass{revtex4-2}
% This looks weird to me, with subsections in italics.
%\documentclass[preprint,1p]{elsarticle}
%\documentclass{llncs}
% The wiley template is a mess. See https://tex.stackexchange.com/questions/701988/failed-to-compile-wiley-njd-v5-latex-template-zip-downloaded-from-wiley-official
%\documentclass{sn-jnl}
% sn-jnl has an external dependency.
%\documentclass[journal=tches,version=final]{iacrj}
%%%%%%%%%%%%%%%%%%%%%%%%%%%%%%%%%%%%%%%%%%%%%%%%%%%%%

\usepackage{tcolorbox}
\newcommand{\BibTeX}{{\rmfamily B\kern-.05em%
   \textsc{i\kern-.025em b}\kern-.08em%
   T\kern-.1667em\lower.7ex\hbox{E}\kern-.125emX}}
\usepackage{verbatim}
\newcommand{\pkgname}{\texttt{metacapture}}
\usepackage{fancyvrb}
\usepackage{tabularx}
\usepackage{xurl}
\newcommand{\iacrcc}{\texttt{iacrcc}}
\newcommand{\cmd}[2][]{%
  \def\FirstArg{#1}%
  \ifx\FirstArg\empty%
    \texttt{\textbackslash{}#2}%
  \else%
    \texttt{\textbackslash{}#2\{#1\}}%
  \fi
}
\newcommand{\pkg}[1]{\texttt{#1}}
\makeatletter
\@ifclassloaded{iacrj}{}{\bibliographystyle{plainurl}
\usepackage{metacapture}
}
\makeatother
\title[plaintext={The metacapture LaTeX package},
       running={The \pkgname\ \LaTeX\ package v\pkgversion},
      ]{The \pkgname\ \textrm{\LaTeX} package v\pkgversion\footnote{Probably not ready for use by others yet.}}

\subtitle[plaintext={Structured metadata from authors}]{Structured metadata from authors}
\addauthor[orcid   = {0000-0003-1010-8157},
           inst    = {1},
           onclick = {https://www.joppebos.com},
           footnote={Authors are allowed to have footnotes on their name.},
	   email   = {joppe.bos@nxp.com},
	   surname = {Bos}
          ]{Joppe W. Bos}
\addauthor[orcid   = {0000-0001-7890-5430},
           inst    = {2},
           footnote = {Kevin is starting to like \pkg{LaTeX3} programming.},
           email   = {latex@digicrime.com},
%           onclick = {https://swcp.com/\%7Emccurley/index.html\#humor},
	   surname = {McCurley},
          ]{Kevin S. McCurley}
\addaffiliation[ror         = {031v4g827},
                street      = {Interleuvenlaan 80},
                city        = {Leuven},
                postcode    = {3001},
                country     = {Belgium},
                countrycode = {BE}
               ]{NXP Semiconductors}
\addaffiliation[country={United States},
                %                countrycode={US}, we omit this.
                department={Department of Redundancy Department},
                state={California},
                city={San Jose}]{Unaffiliated}

\addfunding[country={United States}]{IACR}
\addkeywords[Metadata, publishing, LaTeX]{Metadata, publishing, \LaTeX}
\license{CC-BY-4.0}
%\license{CC0-1.0}
\makeatletter
\@ifclassloaded{iacrj}{\genericfootnote{We can add generic footnotes with \texttt{iacrj.cls}.}}{}
\makeatother
\usepackage{todonotes}
\usepackage{framed}
\newcommand{\todok}[1]{\todo[inline,color=green!20]{K: #1}}
%%%%%%%%%%%%%%%%%%%%%%%%%%%%%%% end of preamble %%%%%%%%%%%%%%%%%%%%
%\RequirePackage[mathlines]{lineno}
%\def\linenumberfont{\normalfont\tiny\sffamily\color{gray}}
%\AtBeginDocument{\linenumbers}
%%%%%%%%%%%%
%\addbibresource{metadoc.bib}
\begin{document}
\hypersetup{colorlinks=true}
\maketitle
\renewenvironment{abstract}{\begin{quote}\noindent\textbf{\textsf{Abstract.}}}{\end{quote}}
\begin{abstract}
  This document describes the \pkgname\ \LaTeX\ package that can
  be used to capture metadata during the compilation of a \LaTeX\
  document. This package is intended for use by document class
  designers as part of a journal publishing workflow. Authors provide their
  title, author information, affiliation, license, etc in macros that
  are used to produce the final document as well as a
  machine-parseable external text file.  This external text file can
  then be used in a publishing workflow to provide HTML pages, JATS, and
  registration with indexing agencies like crossref. This packages comes with
  several implementations of a default \cmd{maketitle} command that
  can be invoked from the package at load time, or the document class
  designer can design their own using documented internal variables of
  the package.
\end{abstract}
\begin{textabstract}
  This document describes the metacapture LaTeX package that can
  be used to capture metadata during the compilation of a LaTeX
  document. This package is intended for use by document class
  designers as part of a publishing workflow. Authors provide their
  title, author information, affiliation, license, etc in macros that
  are used to produce the final document as well as a
  machine-parseable external text file.  This external text file can
  then be used in a publishing workflow to provide HTML pages, JATS, and
  registration with indexing agencies like crossref. This package also
  provides several implementations of a default maketitle command that
  can be invoked from the package at load time, or the document class
  designer can design their own using documented internal variables of
  the package.
\end{textabstract}

%% \ExplSyntaxOn
%% HERE: \g_metac_displayemails_tl
%% \ExplSyntaxOff

\section{Motivation}
The original goal of \TeX\ was focused on typesetting and
the appearance of the output on paper. With the later invention of
\LaTeX, Lamport advised authors that
\begin{quote}As you are writing your document, you should be concerned with its logical
structure, not its visual appearance. The \LaTeX\ approach to typesetting
can therefore be characterized as \emph{logical design}.~\cite[\S 1.4]{latex}
\end{quote}
Users were encouraged to use high-level macros like \cmd{section}, and
leave the decisions like how much space to put before or after a
section to the style that is used.  This separation between structure
and appearance is an example of a more general concept from computer
science known as ``separation of concerns''. The goal of the \pkgname\
package is to extend this concept to metadata about a publication.
Authors provide their metadata (e.g., title, subtitle, keywords,
license, author names, emails, ORCID, funding, affiliations, etc)
without any styling, and the display of this is left completely to the
document class.

We should mention that we insist on metadata consisting of only text
elements, and not \LaTeX\ macros. We allow some simple things like
accents \verb+\"u+ to slip through because they are easily converted
to just text in a post-processing step. We also allow mathematics
inside titles and abstracts. One of the purposes for the \pkgname\
package is to check that authors comply with the restriction to avoid
macros in their metadata. Authors are still able to use macros and
stylized text in their titles, subtitles, and abstracts, but we
require authors to also supply ``plain text'' versions of all
metadata.

\subsection{Previous approaches}
The \cmd{author} macro represents a fundamental limitation of the
original \LaTeX\ \pkg{article} class, because authors are asked to include
formatting as part of their author list using newline codes and
macros such as \cmd{and} and \cmd{thanks}.
If you peek under the covers, the default implementation of the \cmd{and}
macro used to separate authors in the \cmd{author} macro
is given in \texttt{latex.ltx} as
\begin{Verbatim}[samepage=true]
\DeclareRobustCommand\and{%   % \begin{tabular}
  \end{tabular}%
  \hskip 1em \@plus.17fil%
  \begin{tabular}[t]{c}}%     % \end{tabular}
\end{Verbatim}
 The
\cmd{and} macro therefore ends up serving two purposes, namely as a delimiter
between author markup blobs and as a spacing instruction. This clearly
violates the separation of concerns principle because it mixes
structure and appearance. Some document classes such as \texttt{acmart} and
\texttt{llncs} redefine the \cmd{and} macro for other purposes.

In almost all cases, authors need to associate other metadata elements with their
name, such as email, affiliations, ORCID, funding, etc. Depending on which document
class they use, authors have various choices
for this such as the \cmd{thanks} macro to create footnotes, or things
like the \cmd{orcidlink} macro of the \texttt{orcidlink} package. Both of these
are implemented as visual display macros, which again violates the separation
of concerns principle.

More modern document classes
have recognized the need for metadata to be associated with articles
and authors, and each of them has invented their own way to encode
this data.  The \pkg{llncs} class extends \pkg{article} and still uses a
single \cmd{author} macro with authors separated by \cmd{and},
but intersperses other macros like \cmd{orcidID}
and \cmd{inst} inside the \cmd{author} macro to annotate the
individual authors. The implementation of the \cmd{inst} and
\cmd{orcidID} macros are still based on layout rather than structure.
The \pkg{IEEEtran} class also takes this approach.

The \pkg{acmart}, \pkg{amsart}, and \pkg{revtex4-2} document classes
all use a sequence of \cmd{author} commands for each author, with
intervening macros such
as \cmd{orcid}, \cmd{affiliation}, \cmd{email}, etc.\ to describe the
metadata for each author.  The \pkg{acmart} package also defines a
\cmd{additionalaffiliation} macro in case the layout of affiliations
takes too much space, but this places the burden of layout back on
the author instead of the document class.

There have been several packages such
as \pkg{titling}\footnote{See \url{https://ctan.org/pkg/titling}}
and \pkg{authblk}\footnote{See \url{https://ctan.org/pkg/authblk}}
that offer some flexibility in how authors provide their metadata, but
none of them are sufficiently detailed for modern metadata
requirements.

\section{Standard metadata schemas in publishing}
There has unfortunately been no effort among \LaTeX\ document classes
to standardize the syntax for entering metadata, or even which fields
to associate with an author.\footnote{For example, the \pkg{llncs}
class associates emails with affiliations instead of authors.}
This is annoying for authors who try to adapt their \LaTeX\ from one
publisher format to another. Moreover, the metadata associated with authors
and articles has become increasingly complicated over the years, with
new requirements to identify authors by a unique ID (ORCID), as well
as the need to identify institutions by their ROR ID\footnote{See \url{https://ror.org/}}
and a need to identify funding sources with standard identifiers like the
funder ID.\footnote{See \url{https://www.crossref.org/services/funder-registry/}}

In the world of scholarly journal publishing, there have been several efforts to
standarize schemas for metadata. One of the best examples
of this is the schema used by \texttt{crossref.org} for requests to
register a DOI.~\footnote{See
\url{https://data.crossref.org/reports/help/schema_doc/5.4.0/index.html}}
Another well-designed schema is described in the Journal Article Tag
Standard
(JATS)\footnote{See \url{https://jats.nlm.nih.gov/publishing/tag-library/1.4/}}
that is used as a structured document format by many publishers. We
took our guidance from these two schemas in how to represent metadata
in a \LaTeX\ document. In fact, our workflow creates both the crossref
format and the \texttt{<front>} and \texttt{<back>} sections of a JATS
document.  This package does not attempt to cover all possible
metadata associated with an article, but see
section~\ref{missing}. Authors and affiliations are listed
independently, with an \texttt{inst} argument for an author to indicate which
affiliation is associated to an author. Funding is associated to the
document itself rather than the author, in keeping with the schema
provided by \texttt{crossref.org}.

\section{Our solution}
The processing of metadata really has four parts to it:
\begin{enumerate}
\item \label{supply}Authors use \LaTeX\ macros to supply their metadata in a well-structured format.
\item \label{markup}When compiled, the metadata is used to perform visual markup of
the front matter (e.g., title, authors, affiliations, keywords, etc).
\item \label{extract}When the article is published, the metadata is {\em extracted}
from the author-supplied document and used in the publishing workflow. This author-supplied
metadata is combined with publisher-supplied metadata such as volume number,
issue number, dates, etc.
All of the metadata can then be registered with indexing agencies and used to supply 
structured data for the journal web pages and later harvesting agents.
\item \label{xmp} The metadata may be embedded into the output PDF or HTML.
\end{enumerate}
This package addresses all four steps but is primarily focused on steps~\ref{supply}~and~\ref{extract}.
Step~\ref{markup} is typically left to the document class to dictate the style,
but we provide several partial solutions via custom \cmd{maketitle} implementations
that can be invoked to provide the front matter with any existing document class.

\subsection{Author-supplied metadata}
The primary macros used by authors are \cmd{title},
\cmd{subtitle}, \cmd{addauthor}, \cmd{addaffiliation},
\cmd{addfunding}, \cmd{license},
and \cmd{addkeywords}.  A complete description of these is in
Section~\ref{authorusage}. The author enters only the data with these
macros, omitting all formatting of how authors are to be displayed in
the front matter.  Macros other than accents are forbidden in the
primary argument to \cmd{author}, and in particular \cmd{thanks} is
disabled. This is so that the package can clearly identify the name of
the author. Any attributes to the author such as email are added as
optional key-value pairs (\cmd{thanks} is replaced by a \pkg{footnote}
attribute to \cmd{addauthor}).

In our first implementation of metadata capture~\cite{tugboat}, the
metadata extraction was intertwined with the document
class \texttt{iacrcc}~\cite{iacrcc}. In this \pkgname\ package we have
separated out the macros to capture the metadata from the formatting
of metadata. This completes the separation of metadata capture from
document formatting, and allows document classes to style their documents
however they like.


\subsection{Display of metadata}\label{maketitle}
The document class is responsible for the display of front matter, and
this is often (but not always) done with a custom \cmd{maketitle}
macro. Despite the name, the  \texttt{maketitle} macro is is usually
responsible for display of author information, and sometimes also
responsible for display of abstract, keywords, and license. The
display of front matter can be quite complicated, with authors having
multiple affiliations, authors sharing affiliations, footnotes
attached to titles and authors, etc. For
example, \url{https://arxiv.org/pdf/2210.03375} has hundreds of
authors, 75 affiliations, and 12 footnotes on author names (not
surprisingly, they omit email addresses).  The author metadata is
inherently {\em relational}, with authors related to their
affiliations, and other attributes. These relationships are often
represented visually with footnote structures.\footnote{One problem
with this is that the standard
\cmd{footnote} macro does not work inside boxes that may be used to construct
the front matter. This shows up in some two-column formats because the
title and author names are typically displayed in a block across both columns.
We use the \texttt{footnotehyper} package to overcome this.}
There are numerous common styles
for displaying this information, including listing author affiliations
under each author's name (repeating the information), or using
footnotes to show affiliations for authors, or grouping authors
together for a given institute, or authors ordered in some way (e.g.,
alphabetically or randomly). The \pkg{amsart} class places the
affiliations {\em after} the body of the article as endnotes, and so
does \pkg{OUP-EJ} for \emph{The Economic Journal}.

This document is typeset with the standard \texttt{article} document
class, for which default values of \cmd{@title} and \cmd{@author} are supplied to just
work out of the box with the existing \cmd{maketitle}. The
implementation of a \cmd{maketitle} is usually left to the document
class, but the \pkgname\ package provides implementations of
several \cmd{maketitle} styles that can be invoked with the option
\texttt{maketitle=\textless{style}\textgreater}. 
At present, the styles consist of the following (visual appearance
is displayed in Appendix~\ref{appendix}):
\begin{description}
\item[\texttt{iacrj}] Author names are strung together in a list, with
optional ORCID icons after their names, and footnotes to indicate
which affiliations they belong to. Affiliations are listed
individually under the block of author names.  This is the official
version used by the \texttt{iacrj.cls} document class for IACR
journals. It is similar to the first style of \texttt{elsarticle}. See page~\pageref{iacrj}.
\item[\texttt{acmsmall}] This is similar to the \texttt{acmsmall} style of \texttt{acmart.cls},
with one author per line in a vertical list, with author names in small caps followed
by their affiliations and countries. See page~\pageref{acmsmall}.
\item[\texttt{acmconf}] This is similar to the conference proceedings style
of \texttt{acmart.cls}. Each author is listed in a block with their email and
affiliations underneath their name. Shared affiliations are repeated under each author's name.
See page~\pageref{acmconf}.
\item[\texttt{jems}] Modeled after the style used for the {\em Journal of the European Mathematical Society},
in which author names appear before the title, keywords after the
abstract, and each author has an unnumbered footnote that includes the
affiliation, email, and URL. See page~\pageref{jems}.
\item[\texttt{inv}] A left-aligned style inspired in part by the style
of {\em Inventiones mathematicae} in that it uses blocks of text to display
emails. See page~\pageref{inv}.
\item[\texttt{lipics}] This is modeled after the style of the Dagstuhl \texttt{lipics-v2021} document
class. It shows icons for the author email, homepage, and orcid. See page~\pageref{lipics}.
\item[\texttt{ams}] This is similar to what is used in \texttt{amsart}, namely
a title and author names in small caps, with affiliations listed after
the references. For some reason this style has author footnotes
with \href{https://ctan.math.washington.edu/tex-archive/info/amscls-doc/Author_Handbook_Journals.pdf}{no
footnote mark}, so the footnote has to mention the author to give
context in the footnote. See page~\pageref{ams}.
\end{description}
The visual appearance of these styles can be seen in Appendix~\ref{appendix} at
the end of this document.
There is also a \texttt{sample.tex} file supplied with this package that
can be used to test the combination of these \cmd{maketitle} styles
with various document classes.

With the exception of the \texttt{iacrj} style, none of these
represent the official styles of their respective publishers. These
styles are included to allow authors to choose a preferred style, but
also to demonstrate the flexibility of the schema and
to provide useful examples for document class
designers who wish to to implement their own \cmd{maketitle} using the
internal variables documented in Section~\ref{variables}. We believe
that this should simplify the construction of a \cmd{maketitle} macro,
since the variables hold only metadata without formatting.

\todok{The \texttt{amsart} and \texttt{acmart} document classes consider the
\texttt{abstract} to be part of the front matter, and annoyingly require \cmd{begin}\texttt{\{abstract\}}
to appear before \cmd{maketitle}. If we are trying to provide a
\cmd{maketitle} that is consistent to these, then maybe we should capture the
abstract and output it when \cmd{maketitle} is called. This feels
silly to me.}

There are other metadata elements that may need to be displayed, such
as license, keywords, abstract, etc.
The display of these is up to the
document class. Our document class \texttt{iacrj}
has implementations for visual display of license, abstract, and keywords,
but also things like a volume number, issue number, DOI, Crossmark, etc.
A document class can implement these elements in any manner they wish
using the internal variables from this package that are 
defined in Section~\ref{variables}.

\subsection{Capture of metadata}
When a document that uses the \pkgname\ package is compiled, the
author-supplied metadata is extracted from the \LaTeX\ and written
into a \texttt{.meta} file that is machine-parseable. The extraction
of metadata in a machine-readable format during compilation makes it
easy to build publishing workflow systems around \LaTeX, and this was
a big part of the original motivation for this package. An example of
this was used by the journal {\em IACR Communications in
Cryptology}\footnote{See \url{https://cic.iacr.org/}} and the publishing
pipeline system for this is available as open source.\footnote{Source code available
at \url{https://github.com/IACR/latex-submit} and a demo is
at \url{https://publishtest.iacr.org/}.} One part of that system is
a \href{https://github.com/IACR/latex/tree/main/iacrcc/parser}{python
parser} for the file containing extracted metadata that is written by
the package, but it should be easy to write another parser, because
the extracted metadata has a simplified yaml-like structure. The
structure of this file is described in Section~\ref{metafile} and a
sample is given in Figure~\ref{samplemeta}. For more information on
this workflow system, the reader is referred to~\cite{loweringthecost}.

Most journal production workflows are proprietary and opaque, but it
appears that some use parsing tools to extract the metadata directly
from the \LaTeX\ source. Examples of this include the ACM
workflow\footnote{Extraction tools are mentioned in
\url{https://mirror.math.princeton.edu/pub/CTAN/macros/latex/contrib/acmart/acmart.pdf}}
and the Dagstuhl \LaTeX\
project.\footnote{See \url{https://github.com/dagstuhl-publishing/latex}}. This
approach can be difficult because \LaTeX\ is a full programming
language, and things like \cmd{ifx} conditionals make it difficult to reliably parse \LaTeX.
This is one reason why we decided to use \LaTeX\ itself to
produce the metadata in an external file. The only real parser for \TeX\ is the \TeX\ binary
itself, but our approach avoids the problem. It appears that the \texttt{aomart.cls}
document class used for the Annals of Mathematics also follows the approach
of writing metadata to an external file.

\subsection{Embedding metadata in PDF}\label{pdfmetadata}
There have been multiple attempts to provide packages for embedding
metadata into PDF. These include the \texttt{hyperxmp}, \texttt{pdfx},
and \texttt{xmpincl} packages.  The \LaTeX\ team is working on
providing XMP metadata in the PDFs as part of their accessibility
initiative~\cite{xmpinlatex}, and we expect this to be the eventual
solution.  We plan to support this as part of \pkgname\ when the API
for the
\href{https://ctan.math.washington.edu/tex-archive/macros/latex/contrib/pdfmanagement-testphase/l3pdfmeta.pdf}{\texttt{l3pdfmeta}} module becomes stable. Similar solutions should exist to inject the structured
metadata into other output formats such as HTML or EPUB.

We don't require the \pkg{hyperref} package to be loaded unless
the \texttt{maketitle} package option is used or the \cmd{license}
macro is used. If the \pkg{hyperref} package is loaded, then the
\pkgname\ package will set the PDF metadata for \texttt{pdftitle},
and \texttt{pdfkeywords}. If the \texttt{anonymous} option is not used, then
it will also set \texttt{pdfauthor}. If \pkg{hyperref} is loaded, it
should not be loaded with the \pkg{pdfusetitle} option.

\section{Options for loading}
The \pkgname\ package may be loaded by the document class but may also be loaded by the
author. In any event, the \pkgname\ package must be loaded before the author specifies
author, title, etc. \pkgname\  may be loaded with various options:
\begin{description}
\item[\texttt{maketitle=\textless{style}\textgreater}]
If this is used, then the package provides
a \cmd{maketitle}. The \texttt{style} can be any of the styles listed
in Section~\ref{maketitle}.  If this is not chosen, then the class
must define its own \cmd{maketitle} that makes reference to internal variables
of the package. Note that the \cmd{maketitle} macro from \texttt{article.cls} will
work out of the box, because under the covers we implement the \cmd{@title} and
\cmd{@author} macros. This document is typeset using those values. See Section~\ref{maketitle} and
Appendix~\ref{appendix}.
\item[\texttt{anonymous}] If chosen, then the \cmd{maketitle} macro does not disclose
author names or affiliations in the PDF. A document class should load
with this option if it is intending to format for a blind peer review
system.
\item[\texttt{licensereq}] This required the document to specify
a license with the \cmd{license} macro.  At present we only support a
few licenses (see section~\ref{license}) If a document class wishes to
further restrict which license is acceptable, they can check
the \cmd{METAC@license} variable at the end of the preamble.
\item[\texttt{textabstract}] if chosen, then the document must specify a
separate ``text-only'' abstract that is free of macros other than
mathematics in a \texttt{textabstract}
environment that contains no user-defined macros. This abstract is in
addition to the ordinary \texttt{abstract} environment, and results in a file
named \cmd{jobname.abstract} that contains the abstract when the paper is compiled.
We ask for such an abstract from authors so that we can capture an abstract
that is suitable for indexing and HTML pages.
\item[\texttt{emailreq}] this takes one of three possible options \texttt{none,one,all} that indicates whether
no emails are required for authors, at least one email is required for
some author, or all authors must supply an email. This option might be used by
a document class that wishes to require a corresponding author.
\item[\texttt{orcidreq}] whether each author must have an ORCID. Keep in mind that
some authors may refuse to use an ORCID. The ORCID of an author should probably only
be included if it is supplied by the author themself.
\item[\texttt{notitlefootnote}] when selected, the \cmd{footnote} macro is disabled inside
the main argument of \cmd{title}
%\item[\texttt{lefttitle}] when selected with \texttt{maketitle}, the title and authors will be left-aligned
\item[\texttt{footnotesymbols}] Some of the options for \texttt{maketitle} use a different
style of footnote marker for affiliations from the rest of the footnotes. For example, in the
\texttt{iacrj} style the footnotes on title and authors would ordinarily be 
labeled as a,b,c, but they are labeled as symbols
\textasteriskcentered, \dag, \ddag, etc if the \texttt{footnotesymbols} option is also used.
Note that this option should be used with caution, because at most 10 authors
can have footnotes with this option.
\end{description}
\section{Usage by authors}\label{authorusage}
The main macros for authors that are provided by this package
are \cmd{title}, \cmd{subtitle}, \cmd{license}, \cmd{addauthor},
\cmd{addaffiliation}, \cmd{addfunding}, and \cmd{addkeywords}.
These can only be used in the preamble before \cmd[document]{begin}.
There is also a \texttt{textabstract} environment
to capture text-only versions of the abstract.
\subsection{Title}
A title is added using the \cmd{title} macro, which has a number of optional arguments:
\newcommand{\argrow}[2]{\texttt{#1} & #2\\}
\newenvironment{arglist}{%
  \begin{flushleft}\renewcommand{\arraystretch}{1.2}\begin{tabular}{@{}lp{0.7\linewidth}}%
  }{%
  \end{tabular}\end{flushleft}
  }
\begin{arglist}
    \argrow{running}{The running title intended for display in the headers.}
    \argrow{plaintext}{A text version of the title (mandatory if macros are used in the title).}
%    \argrow{footnote}{Add a footnote to the title. Only one is allowed.}
\end{arglist}
\todok{Some document classes override \cmd{title} macro in different ways from the
\pkgname\ package - notably \texttt{amsart} and classes such as \texttt{acmart}
that extend it. \texttt{amsart} accepts two arguments to \cmd{title} and then
defines \cmd{shorttitle} inside \cmd{title}. Without this, you cannot use
the standard \cmd{maketitle} of \texttt{amsart}. I inserted a special case to check of \texttt{amsart}
was loaded and then set \cmd{shorttitle}, but I wouldn't want to see a lot of special cases
like this. Should we call ours \cmd{addtitle} and then disable \cmd{title} the
same way we did for \cmd{author}? }

\noindent An example using all the optional arguments is shown below.

\begin{Verbatim}[samepage=true]
\title[running   = {The iacrcc class},
       plaintext = {How to use the iacrcc LaTeX class},
      ]{How to use the \texttt{iacrcc} \LaTeX\
      class\footnote{A revision of an earlier paper on arxiv.org}}
\end{Verbatim}

The \verb+plaintext+ option is only required if you use macros in
your title (it is required in the example). Inline mathematics and
accents like \verb+\"u+ are allowed in the main argument to \cmd{title},
and so are newlines \texttt{\textbackslash\textbackslash}. Note
that \LaTeX\ has defaulted to UTF-8 input since 2019, so just ü is
preferred to \verb+\"u+. Note also that \cmd{thanks} is disabled
inside \cmd{title}, and \cmd{footnote} can optionally be disabled by loading
\pkgname\ with the option \texttt{notitlefootnote}. See
Section~\ref{footnotes} for information about footnotes.

In our previous implementation from \texttt{iacrcc.cls}, we had a
\texttt{subtitle} attribute, but that has now been moved into a separate
\cmd{subtitle} macro in order to support a plain text version.

\subsubsection{Subtitle}
An author is always allowed to have a two-line title by inserting a
newline \texttt{\textbackslash\textbackslash} into the main argument
of \cmd{title}, but a subtitle would often be typeset in a smaller
font.  The semantics of a subtitle are always a little unclear, but
the most common definition is for a ``subordinate or explanatory
title''.\footnote{The JATS standard states that ``The <subtitle> is a
subordinate or auxiliary title that adds information to the full title
or modifies the full title.''}  If an author wishes to have a
subtitle, they use the \cmd{subtitle} macro, which also requires an
optional \texttt{plaintext} attribute if the main argument
to \cmd{subtitle} contains any macros.  A full example could be:
\begin{Verbatim}[samepage=true]
\subtitle[plaintext={A LaTeX tutorial}]{%
     A \LaTeX\ tutorial\protect\footnote{Thanks to Leslie Lamport}}
\end{Verbatim}
Note that footnotes need to be protected inside a subtitle.
The \texttt{notitlefootnote} option also prevents
\cmd{footnote} from being used inside \cmd{subtitle}. A
document class is free to treat subtitles in any way they see fit, but
if the \cmd{title} macro is used with the \texttt{running} attribute,
then the subtitle should probably not be added to a running title.

\subsection{Authors}
Author information is entered using the \cmd{addauthor},
\cmd{addaffiliation}, and \cmd{addfunding} macros. Authors are asked
to enter this information in a structured way so that we can provide
it to indexing agencies. The \cmd{author} macro is disabled.

Authors are listed individually using repeated calls to
the \cmd{addauthor} command, and these must appear before
\cmd{begin\{document\}}. The \cmd{addauthor} macro has a number of optional
arguments shown in Figure~\ref{addauthor}.
\begin{figure*}
\begin{arglist}
\argrow{inst}{A numerical list of 1-based indices specifying an institution in the 
                     affiliation array (see below).}
\argrow{orcid}{The ORCID of the author, specified using the 19-character
format \texttt{xxxx-xxxx-xxxx-xxxx}.}
\argrow{footnote}{Create an author-specific footnote.}
\argrow{surname}{Indicate the surname of the author for indexing purposes.}
\argrow{onclick}{Provide a URL for the author, e.g., a home page.\todok{I don't like this option, since home page links can always be put into a footnote. A document style could choose to make the author name clickable, but it's not clear if that would lead to the ORCID or the home page.}}
\argrow{email}{Define the e-mail address of this author. Note that the load
   option \texttt{emailreq} may place restrictions on whether an author needs
   to supply an e-mail address.}
\end{arglist}
\caption{Arguments to \cmd{addauthor}}
\label{addauthor}
\end{figure*}

The display of these elements by a document class may be customized in
any way the document designer sees fit. In some of the \cmd{maketitle}
implementations provided by \pkgname, the presence of
the \texttt{orcid} attribute creates a small clickable orcid logo next
to the authors name looking like \orcidlink{0000-0003-1010-8157} that
is a hyperlink to the authors ORCID
(see: \url{https://orcid.org}). Similarly, some of our implementations
of \cmd{maketitle} display the \texttt{onclick} attribute with an icon
like \linkicon\ or \homelink{https://iacr.org}
displayed next to the author's name that is an active
link to the URL.

It's not obvious how to interpret the omission of the \texttt{inst} argument from
\cmd{addauthor}.
It's possible that the author has no affiliation, but it's also possible that the author
is affiliated with all listed affiliations. That is a matter of policy for the document
class. In order to eliminate this ambiguity, the document class may choose to require the
\texttt{inst} argument for every author, and use an empty \texttt{inst} argument in case
the author has no affiliation. In the \cmd{maketitle} implementations supplied in this
package, we have chosen to omit the footnotes on author names for affiliations in the following cases:
\begin{itemize}
\item if \cmd{addauthor} omits the \texttt{inst} attribute or it is empty,
\item if there is only a single author,
\item if there is only a single affiliation
\end{itemize}
In the last two cases we also omit the numbers on the affiliations. The \texttt{inst}
array serves two purposes, namely for appearance to link authors to affiliations, and
for metadata processing in a journal workflow where author affiliations are reported.
In the latter case the indices must be validated to make sure that they refer to actual
entries in the affiliation array.

Some downstream processors like crossref request author names to be broken
into \texttt{given-name,surname} but this is in conflict with many existing
cultural norms for author names (see~\cite{falsehoods}).
\texttt{crossref} has a required element for
surname, which is why we include this.
%% \todok{We recently had a case in CiC
%% with the author name ``Arthur Herlédan Le Merdy'' in which their
%% surname is ``Herlédan Le Merdy''. In this case the bibtex parser
%% failed, the python bibtex parser failed, and the HumanName parser
%% failed to identify the surname. Another example was an author named
%% ``Mahdi Rahimi'', where the HumanName parser failed to recognize
%% ``Mahdi'' as a given name. It is simply not feasible to reliably parse
%% names and recognize what the given name and surname should be, and
%% there is no real reason to require it other than alphabetic ordering
%% on author names.}

When the URL provided to the {\texttt onclick} option contain characters
with a ``special'' meaning in \LaTeX{} they might render incorrectly.
For example, the URL
\begin{quote}
    \verb+https://web.com/~foo/the best/#zoo+
\end{quote} contains
a tilde, a space, and a pound symbol \#. It would
be encoded as 
\begin{verbatim}
  onclick = {https://web.com/\%7Efoo/the\%20best\#zoo}
\end{verbatim}
\todok{I don't think we really need the \texttt{onclick} attribute, since these can be
included in footnotes where the URL is displayed in print.}

An example using all the optional arguments is given below. In this case
the author has \verb+inst={1,2}+ to indicate that they are affiliated with
the first and second affiliations that are entered with
\cmd{addaffiliation}:

\begin{Verbatim}[samepage=true]
\addauthor[orcid    = {0000-0000-0000-0000},
           inst     = {1,2},
           footnote = {Thanks to my supervisor for the support.},
           onclick  = {https://www.mypersonalwebpage.com},
           email    = {alice@accomplished.com},
           surname  = {Accomplished},
          ]{Alice Accomplished}
\end{Verbatim}

\todok{It might be better to use explicit \texttt{label=abc} in \cmd{addaffiliation} and
then use \texttt{inst={abc}} in \cmd{addauthor}. This would remove any dependency on
the order in which affiliations are listed in the file, so if an author is added then we
wouldn't have to renumber the affiliation references. Unfortunately it would require us to
have a two-pass process to generate the footnote number.}
The \cmd{thanks} macro is disabled
inside \cmd{addauthor}, so use the \verb+footnote+ option
on \cmd{addauthor} instead. In fact, if an author attempts to use any
non-accent macros inside the primary argument to \cmd{addauthor} it
generates an error.
\subsection{Affiliations}
Affiliations are listed individually using the \cmd{addaffiliation} command
\emph{after} the last author has been added using \cmd{addauthor}. It can
only be used before \cmd{begin\{document\}}, and has several optional arguments:

\begin{arglist}
\argrow{ror}{The Research Organization Registry (ROR) indentifier
             for this affiliation. This is the equivalent of ORCID for organizations. See \url{https://ror.org/}.}
\argrow{department}{Department or suborganization name.}
\argrow{street}{Street address.}
\argrow{city}{City name.}
\argrow{state}{State or province name.}
\argrow{postcode}{Zip or postal code.}
\argrow{country}{Country name. This is strongly recommended.}
\argrow{countrycode}{ISO-3166 Alpha-2 identifier for country. This is strongly recommended, and
it eliminates ambiguity in country name
(e.g., Österreich vs Austria). If \texttt{country} is omitted, this can be used to fill it in.
A list of these can be found at \url{https://en.wikipedia.org/wiki/ISO_3166-1_alpha-2}.}
\end{arglist}
\noindent There is an online tool at
\href{https://publish.iacr.org/funding}{\texttt{publish.iacr.org/funding}}
to help you find ROR identifiers, and authors are strongly urged to
include these.
It is up to the implementation of \texttt{maketitle} to decide whether to show all
attributes on an affiliation. Most implementations will use the name, \verb+city+ and
\verb+country+ arguments. All arguments can be used to provide metadata
to indexing agencies.

A full invocation of \cmd{addaffiliation} would look like:
\begin{Verbatim}[samepage=true]
\addaffiliation[ror        = {05f950310},
                department = {Computer Security and Industrial Cryptography},
                street     = {Kasteelpark Arenberg 10, box 2452},
                city       = {Leuven},
                state      = {Vlaams-Brabant},
                postcode   = {3001},
                country    = {Belgium},
                countrycode = {BE}
               ]{KU Leuven}
\end{Verbatim}
\subsection{Funding information}
Authors should use the \texttt{\textbackslash addfunding} macro to
make sure that funding agencies can find articles published under their
sponsorship. An example is:
\begin{verbatim}
\addfunding[fundref = {100000001},
            grantid = {CNS-1237235},
            country = {United States}]{National Science Foundation}
\addfunding[ror     = {00pn5a327},
            country = {United States}]{Rambus}
\end{verbatim}

\noindent In this example, the author acknowledges a grant from the
National Science Foundation and support from Rambus (with no
\texttt{grantid}). The inclusion of funding from an agency without a
\texttt{grantid} might be appropriate if the author simply received
support for a visit.

The complete list of optional arguments for \texttt{\textbackslash addfunding} is:
\begin{arglist}
\argrow{fundref}{An identifier from the
                    \href{https://publish.iacr.org/funding}{Crossref funder registry}.}
\argrow{ror}{An identifier from the 
                    \href{https://publish.iacr.org/funding}{Research Organization Registry} 
                    (ROR).}
\argrow{country}{The country of the funding agency.}
\argrow{countrycode}{ISO-3166 Alpha-2 identifier for country.}
\argrow{grantid}{The identifier of the grant that is assigned by the agency 
                    who provided it.}
\end{arglist}

\noindent You can use the online tool at 
\href{https://publish.iacr.org/funding}{\texttt{publish.iacr.org/funding}} to
help you find \texttt{fundref} and \texttt{ror} identifiers.

Note that \cmd{addfunding} \textbf{does not} automatically create footnotes or
an acknowledgements section to identify funding - it only collects the
metadata for indexing. If you wish to include such visible
annotations, you can use the \texttt{footnote} option on
\cmd{addauthor} or add a separate
acknowledgements section. Some funding agencies have specific
requirements for how they want to be acknowledged in the article.
\todok{It occurs to me that we could include an optional \texttt{text} attribute
that is optionally displayed by the document class either as a footnote or in an
acknowledgements section.}

\subsection{Footnotes}\label{footnotes}
Authors may be accustomed to using \cmd{thanks} for footnotes
indicating affiliation, email, or funding, but the
\cmd{thanks} macro is disabled and authors should use the methods described
in this document.  We provide the \texttt{footnote} attribute on
authors so that they can add an arbitrary footnote to their name.
This can be used for indicating that the author's affiliation for the
work was different than their current affiliation, or to indicate
contact address, or a previous name, etc.  Some of the implementations
of \cmd{maketitle} use footnotes to connect authors to their affiliations.
Document designers often have specific requirements on footnotes, and one such
requirement is supported by the \texttt{notitlefootnote} option of
this package in case footnotes are not allowed on titles.

It should be noted that footnotes are specifically tied to paper-oriented
layouts, and can be problematic in HTML output.
\subsection{License}\label{license}
When the \texttt{licensereq} option is used upon load, the author needs
to provide a supported license.   At present the only
acceptable licenses are the following creative commons licenses:
\texttt{CC-BY-4.0},
\texttt{CC-BY-NC-4.0},
\texttt{CC-BY-NC-ND-4.0},
\texttt{CC-BY-NC-SA-4.0},
\texttt{CC-BY-ND-4.0},
and \texttt{CC0-1.0}.
An example would look like:

\begin{verbatim}
\license{CC-BY-4.0}
\end{verbatim}

\subsection{Keywords}
Use \cmd{addkeywords}\{keyword1, keyword2\} to give a
list of keywords or key phrases. This is an optional macro that should
appear before the abstract.  Individual keywords should be separated
by commas. If the keywords contains math or macros, then you must supply an additional set of
text-only keywords; for example:
\begin{verbatim}
\addkeywords[rings, arithmetic on Z]{
   rings, arithmetic on $\mathbb{Z}$}
\end{verbatim}
\subsection{Abstract}
A document class that loads the \pkgname\ package may format the
abstract however it is desired, but \pkgname\ also provides a
mechanism for extracting a ``text-only'' abstract. If the author
provides such an abstract within the \texttt{textabstract}
environment, it will create a file
named \texttt{\textbackslash{jobname}.abstract} that contains the
contents.  The purpose of the text-only abstract is to provide for
indexing and production of {HTML} pages to describe the paper. As
such, it is just as important as the classical \texttt{abstract} of a
paper because it contains a textual summary that readers will use to
decide if the paper is worth reading. The only difference is that the
contents of the
\texttt{textabstract} is constrained on what it may contain.

Note that the contents of the \texttt{textabstract} will not be
displayed in the final PDF except as metadata. Note also that
\verb+\begin{textabstract}+ must appear on a line by itself.

%% \section{Auxiliary files}
%%   Users will already be familiar with the fact that running a latex compiler
%%   will produce a number of auxiliary files, including the \texttt{.log},
%%   \texttt{.aux}, \texttt{.bbl}, \texttt{.blg},  \texttt{.toc}, and
%%   \texttt{.out} files produced by \texttt{bibtex}
%%   and the \texttt{hyperref} package.  If the main \LaTeX\ file is \texttt{main.tex},
%%   then the \pkgname\ package will produce two additional files, namely
%%   \texttt{main.meta} and \texttt{main.abstract}. The \texttt{main.meta}
%%   contains all metadata from the paper, and the file \texttt{main.abstract} contains
%%   the contents of the \texttt{textabstract} environment.

\section{Format of the \texttt{.meta} file}\label{metafile}
The \texttt{metadoc.meta} file that is created when a \LaTeX\ document
is compiled is similar to yaml. An example is shown
in Figure~\ref{samplemeta}.
\begin{figure*}
\begin{tcolorbox}[colback=blue!2!white,boxrule=.7pt]
\begin{verbatim}
schema:0.9
title: The metacapture LaTeX package
  subtitle: A demo with different styles and classes
author:
  name:Paul Erdős
  orcid:0000-1111-2222-3333
  inst:1,2
  footnote:Paul has a footnote
  email:erdos@att.com
  surname:Erdős
author:
  name:P\'al Tur\'an
  orcid:0000-0001-7890-5430
  inst:3
  footnote:Another remarkable Hungarian mathematician
  email:latex@digicrime.com
  surname:Tur\'an
affiliation:
  name:University of California, San Diego
  ror:0168r3w48
  department:Computer Science Department
  country:United States
affiliation:
  name:Mega Corporation
  department:Department of Redundancy Department
  city:Sunnydale
  state:California
  country:Elbonia
affiliation:
  name:Faber College
  country:Absurdistan
  department:Department of Unfundable Research
  city:Gottaknow
keywords: Metadata, publishing, LaTeX
license: CC-BY-4.0
\end{verbatim}
\caption{Sample \texttt{.meta} file that is described in Section~\ref{metafile}.
The \texttt{schema} attribute indicates a version of \pkgname\ that
was used to create the file. The resst of the format should be fairly
clear.}
\label{samplemeta}
\end{tcolorbox}
\end{figure*}
While this looks like yaml, it's not quite the same. The reader
might wonder why we don't write yaml, and the real reason is that yaml
requires enclosing strings inside double quotes if they contain any of the characters
\verb+{}[]&*#?|-<>=!\%@:+, and those characters would need to be escaped.
This would be a pain to implement in \LaTeX, and we don't need the
full generality of yaml.  The syntax of the \texttt{.meta} file is
simplified by the fact that every value is on a single line.

\section{Internal variables}\label{variables}
\todok{This section needs review since the package has been converted to \texttt{expl3}.}
For those seeking to implement their own document class based on this, you should
make use of some internal variables.
If a document class wishes to provide additional restrictions on the
metadata that is provided, then they can implement additional checks
on these variables at the end of the preamble. An example might be to
check that every author supplied a surname, or that every author supplied
an affiliation.

The most important internal variables are listed in
Table~\ref{othervariables}. We believe that these are sufficient to
construct any form of front matter that is desired, and we provide
several implementations of a \cmd{maketitle} command that can be
accessed through the load
option \texttt{maketitle=\textless{style}\textgreater}. 

The first version of this package was written in LaTeX2$\epsilon$
syntax, but that made it complicated to store a list of authors,
affiliations, or funding agencies.  The \pkgname\ package is now
implemented using functionality from the LaTeX3 programming
layer.\footnote{For those who are unfamiliar with this, we recommend
reading \url{https://ctan.math.washington.edu/tex-archive/macros/latex/required/l3kernel/expl3.pdf}
and the reference
manual \url{https://ctan.math.washington.edu/tex-archive/macros/latex/required/l3kernel/interface3.pdf}.}
In particular, this means that some variable names follow the
general pattern of\\
\texttt{\textbackslash\textless{scope}\textgreater\_\textless{module}\textgreater\_\textless{name}\textgreater\_\textless{type}\textgreater},
where
\begin{itemize}
\setlength\itemsep{0pt}
\item \texttt{\textless{scope}\textgreater} is either \texttt{g} or \texttt{l} for global or local variables,
\item \texttt{\textless{module}\textgreater} is the string \texttt{metac}, which we use to denote the module,
\item \texttt{\textless{name}\textgreater} is a variable name,
\item \texttt{\textless{type}\textgreater} is a data type.
\end{itemize}
The two most important data types from the LaTeX3 programming layer
are the \texttt{seq} and \texttt{prop} data
structures. The \texttt{prop} data structure is a property list, and
is much like a dictionary that holds key-value pairs.  This is a
natural match for storing each author, which is itself a set of
key-value pairs.  The same goes for each affiliation and each funder.
The other important data structure is \texttt{seq}, which is a
sequence. We use the variable
\cmd{g\_metac\_author\_seq} to hold the sequence of authors.
Due to a limitation of the \texttt{seq} and \texttt{prop} objects, the
sequences hold only serialized versions of the author \texttt{prop}
rather than the \texttt{prop} object itself.\footnote{Apparently the
entry of a \texttt{seq} variable can only be ``balanced text'' as
defined in the \TeX\ book.
See \url{https://tex.stackexchange.com/questions/115700/can-i-store-sequences-in-sequences-with-expl3}
and \url{https://github.com/latex3/latex3/issues/500} where the LaTeX
team discussed such nested data structures and decided not to support them.}
Finally, there is an additional datastructure called a \texttt{clist}
for comma-separated list that is useful for holding the lists of
keywords.

If any of the LaTeX3 variables are used in a document class, then the
code has to be enclosed
inside \cmd{ExplSyntaxOn}...\cmd{ExplSyntaxOff} groups. This is not a
serious limitation, since it's much like the restriction to access
variables that contain the \texttt{@} character
inside \cmd{makeatletter}...\cmd{makeatother} blocks.

\newcommand{\vardesc}[2]{\item[#1]\hfill\\#2}
\begin{table*}\label{othervariables}
\begin{tcolorbox}[colback=blue!5!white,boxrule=0.7pt]
\begin{description}
\setlength\itemsep{0pt}
\vardesc{\cmd{g\_metac\_author\_seq}}{the list of authors, each of which is a serialized key-value \texttt{prop}}
\vardesc{\cmd{g\_metac\_affil\_seq}}{the list of affiliations}
\vardesc{\cmd{g\_metac\_funders\_seq}}{the list of funders}
\vardesc{\cmd{g\_metac\_keywords\_raw\_clist}}{The list of raw encoded keywords (may contain macros)}
\vardesc{\cmd{g\_metac\_keywords\_plaintext\_clist}}{The list of plaintext keywords}
\vardesc{\cmd{METAC@license}}{When \cmd{license} is called, this is set to the license identifier.
This is an SPDX identifier because of our dependence on the
\texttt{doclicense} package. An example is \texttt{CC-BY-4.0}.}
\vardesc{\cmd{if@metacapture@anonymous}}{Set if the anonymous option is used to load it.}
\vardesc{\cmd{METAC@displayemails}}{This is a comma-delimited list of \texttt{email,(name)} values
that were constructed from calls to the \cmd{addauthor} macro.}
\vardesc{\cmd{@title}}{The formatted title supplied by the author as argument \texttt{\#2}
of \cmd{title}. This does not include anything from \cmd{subtitle}.}
\vardesc{\cmd{METAC@title@running}}{Optional running title supplied by the author.}
\vardesc{\cmd{METAC@title@plaintext}}{Optional plain text title.}
\vardesc{\cmd{METAC@title@footnote}}{Optional footnote for the title.}
\vardesc{\cmd{METAC@subtitle}}{Optional subtitle.}
\vardesc{\cmd{METAC@subtitle@plaintext}}{Optional plain text version of subtitle.}
\vardesc{\cmd{METAC@listofauthors}}{A list of author names separated by ', '}
%\vardesc{\cmd{@author}}{A marked up list of authors that is used internally by the \cmd{maketitle} of the package.}
\vardesc{\texttt{METAC@author@cnt}}{A counter for the number of authors. It is incremented each time \cmd{addauthor} is called.}
\vardesc{\texttt{METAC@email@cnt}}{A counter for the number of authors with email.}
\vardesc{\texttt{METAC@affil@cnt}}{A counter for the number of affiliations. It is incremented each time \cmd{addaffiliation} is called.}
\end{description}

\end{tcolorbox}

\caption{Internal variables that are set by calls to \cmd{addauthor}, \cmd{addaffiliation}, \cmd{addfunder},
\cmd{addkeywords}, \cmd{title}, \cmd{subtitle}, and \cmd{license}. Some of these are LaTeX3-specific, as indicated
by the name used for them. All of these are available at the end of the preamble, because the commands to set
them may only be used in the preamble.}
\end{table*}

A complete tutorial on the use of \texttt{expl3} is beyond the scope
of this article, but we hope that the source code of the package
contains sufficiently many examples of how to use the variables.

\section{What's missing}\label{missing}
The purpose of this package is to capture author-supplied metadata rather
than publisher-supplied metadata such as a DOI or page numbers. Such
publisher-supplied metadata is often encoded into the PDF of a
publication, e.g.\ as a hyperlink to the DOI. We leave the handling of
publisher-supplied metadata to the document class, but the \pkg{iacrj.cls}
and our open-source workflow may prove useful as an example.

The breadth of metadata for a publication has been growing in recent years.
We have attempted to include only the minimal metadata elements that have clear
definitions and are currently required in all disciplines, but we fully expect
that others may be needed in the future. Some things include:
\begin{description}
\item[licenses] We currently only support a limited selection of licenses (e.g., we omit
copyleft). It's possible that someone may wish to place different licenses on media embedded
in the document. It's also possible that someone may wish to place different licenses on the \LaTeX\ source
than the final document intended for readers. We do not cover these cases.
\item[languages] We have no way for an author to express which languages are used in the document,
or to provide language-specific versions of title, keywords, affiliation name, abstract, etc.
\item[affiliations] There are a number of other elements that might be associated with an affiliation,
including address lines for a postal address, phone number, a URL, or other identifiers such as
Grid, Ringgold, Scopus, etc.

\item[XMP] XMP stands for ``eXtensible Metadata Platform'', and is an XML standard for
embedding metadata into PDF as well as other document
formats. Unfortunately the schema lags badly behind other standards
(it doesn't even have support for ORCID without resorting to
non-standard extension schema). See Section~\ref{pdfmetadata}.
\newcommand\CREDIT{CRediT}
\item[Contributor roles] There have been various attempts to define a taxonomy of roles
played by authors. The \texttt{amsart.cls} document class allows specifying a {\em contributor} with \cmd{contrib}
and a {\em role} argument to say things like ``with an appendix by N.\ Bourbaki'' after the list of authors.
They do not appear to report this information to crossref.
Perhaps the best known definition of contributor roles is \CREDIT, which stands for
Contributor Role Taxonomy, and has now become an ISO
standard.\footnote{See \url{https://credit.niso.org/}} Crossref has
announced that they will support something like this in version~5.5 of their schema.
\todok{The only identifiers for \CREDIT\ roles include things with spaces in them like \texttt{Funding acquisition}. We could add something like \texttt{credit=role} to author, but
I'm worried that it isn't sufficiently expressive yet. In particular some people have complained
\texttt{translation} is omitted. It also omits the degree of contribution in each role.}
\item[Bibliographic references] Since most users of \LaTeX\ use \texttt{bibtex} or \texttt{biblatex},
it is natural to think of exporting bibliographic references as a
structured part of the metadata for the article. There are several
problems with this, including the fact that the fields for a \BibTeX\
entry are not well defined and the format has failed to evolve.\footnote{The
original \BibTeX\ documentation says ``Don't take the field names too
seriously''.}  For example, authors may add things like a URL as part
of the \texttt{url} field, or a \texttt{note} field, or
a \texttt{howpublished} field.  Moreover, packages
like \texttt{biblatex} have added additional entry types and fields.

Given the weaknesses of the \BibTeX\ format, we might consider an
alternative export format.  There are several such bibliographic
database formats, but they are seldom used with \LaTeX, and they all
suffer from deficiencies.  These include
RIS,\footnote{See \url{https://en.wikipedia.org/wiki/RIS_(file_format)}},
Endnote,
Zotero,\footnote{See \url{https://gist.github.com/pchemguy/19fa69fb4e74ef0cca0026aa0dbf5f42}},
citeproc
JSON,\footnote{See \url{https://github.com/citation-style-language/schema}}
and
JATS.\footnote{See \url{https://jats.nlm.nih.gov/publishing/tag-library/1.4/element/element-citation.html}}

In our first effort at metadata extraction~\cite{tugboat}, we used a
custom \BibTeX\ style to export the bibliography in a structured
format, but that introduced additional problems because we wanted to
follow the separation of concerns principle.  In the end we decided
that exporting bibliographic references is a big complicated mess that
is better left to a high-level language.  In our companion workflow
software,\footnote{See \url{https://github.com/IACR/latex-submit}} we
use python to invoke \pkg{bibexport} to find the cited references, and
then parse the bibtex files directly. This was complicated by the fact
that we wanted to support both
\pkg{biblatex} and \pkg{bibtex}.

\item[Name parts] Some agencies like crossref are attempting to gather names of authors in
two parts, namely first and last (or given and family name). We have
attempted to comply by allowing an optional surname field on author
names, but this approach is flawed since names cannot be assumed to
have the same structure across all cultures. See~\cite{falsehoods}.
We also do not support alternate names for authors.
\item[Funding groups] We currently support the name, identifier, and award number for a funder,
but we may wish to provide further information like the name of a PI or the
program within a larger funding organization. This would be driven by downstream requirements.
\item[Shared footnotes] We don't support shared footnotes for authors. These might be useful to
for a single statement that they contributed equally, or to identify all corresponding authors.
We also don't support
multiple footnotes on an author or a title.
\item[Multiple departments] Consider the case where author$_1$ is in the mathematics
department of UCSB, and author$_2$ wishes to list both the mathematics and computer science departments
in their affiliation.  In this case it's not clear how the affiliations should be listed.
One choice is to list UCSB twice, with author$_1$ specifying
the mathematics department and author$_2$ specifying both departments in the \texttt{department}
attribute. Alternatively, the UCSB affiliation would be listed once, but footnotes used on the
author to indicate which department. The \pkg{acmart} document class
has some support for this.
\item[Discipline-specific data] Some disciplines use additional metadata such as
clinical trials that are registered with a International Standard
Randomized Controlled Trial Number (ISRCTN), or the \verb+ClinicalTrials.gov+
number.  We don't understand them well enough to include them
here, but they seem like natural extensions.
\item[External documents] Some journal articles are explicitly linked to
other documents or media. This could include supplementary material,
former versions of the document, translations, related media,
data, code, clinical information, etc. 
\todok{Perhaps we should supply a way to extend the schema, say with a
macro like \texttt{\textbackslash{@addmetafield}\{fieldname\}} that
could be called from a class file, or something more sophisicated
based on \texttt{xkeyval}. This could be used in a document class to
ask for things like clinical trials or domain-specific keywords or
other metadata.  My fear is that
this would not be expressive enough for hierarchical structures or
lists, and we may have to supply domain-specific rules for what values
are accepted.}
\item[Keywords and taxonomies]
Some disciplines also use specific taxonomies or keyword vocabularies (e.g.,
ACM Computing Classification System, AMS Mathematics Subject
Classification, or the JEL classification system in economics).  At
present, we regard these as too publisher-specific be included in this
general package. A document class can always provide support for them.


Both JATS and Crossref have support for keywords and/or subject classifications in
their schemas. In both cases there is support for multiple classifications, with multiple
vocabularies or assigning authorities.
\end{description}
\section{Feedback}
Use the \pkgname\ github project to report bugs and submit feature
requests.\footnote{See \url{https://github.com/IACR/latex/tree/main/metacapture}}
If
your feature is only relevant to a specific discipline, then perhaps
the natural thing to do is to extend the \pkgname\ package and add
additional fields. Adding too many fields and too much complexity can
make the documentation hard to digest.

\todok{Should we upload to CTAN?}

%\printbibliography
\bibliography{metadoc}
\appendix
\section{Appendix: Example styles for \cmd{maketitle}\label{appendix}}
This document was typeset with \texttt{article} class and the
default \cmd{@title} and \cmd{@author} (i.e.,
using \texttt{maketitle=none}).  In subsequent pages we show the
appearance of the different styles for \cmd{maketitle}. A class designer can of
course make their own \cmd{maketitle} to suit their own needs, and hopefully these
examples will be useful.
\makeatletter

\newpage
\phantomsection{}
\setcounter{footnote}{0}
\METAC@iacrj@maketitle
\begin{quote}
This uses \texttt{maketitle=iacrj}. Footnotes for affiliations are numbered, but footnote symbols on
title footnotes and author footnotes are alphabetic (they can also be symbols). The icon for a home page
is different than what is used in \cmd{@author}.
\end{quote}
\label{iacrj}

\newpage
\phantomsection{}
\setcounter{footnote}{0}
\METAC@acmsmall@maketitle
\begin{quote}
This uses \texttt{maketitle=acmsmall}. Author names are in small caps.
\end{quote}
\label{acmsmall}

\newpage
\phantomsection{}
\label{acmconf}
\setcounter{footnote}{0}
\savenotes
\METAC@acmconf@maketitle
\spewnotes
\begin{quote}
This uses \texttt{maketitle=acmconf}. Each author is displayed in a block with repeated affiliations.
It appears similar to the default \cmd{@author}, but the spacing is better for more than a couple of
authors.
\end{quote}

\newpage
\phantomsection{}
\label{jems}
\setcounter{footnote}{0}
\METAC@jems@maketitle
\begin{quote}
This uses \texttt{maketitle=jems}. Author names appear above the title, and each author has an unnumbered footnote with their
information. It's not clear what to do with footnotes on author names, and the journal class appears not to support
them.
\end{quote}

\newpage
\phantomsection{}
\label{inv}
\setcounter{footnote}{0}
\begin{savenotes}
\METAC@inv@maketitle
\end{savenotes}

\begin{quote}
This uses \texttt{maketitle=inv}. Affiliations are listed below each author name, and are repeated for shared
affiliations. Emails are listed after affiliations in a block.
\end{quote}

\newpage
\phantomsection{}
\label{lipics}
\setcounter{footnote}{0}
\METAC@lipics@maketitle
\begin{quote}
This uses \texttt{maketitle=lipics}. Author names have icons for email, home page, and ORCID.
Affiliations are listed below each author name, and are repeated for shared
affiliations.
\end{quote}

\newpage
\phantomsection{}
\label{ams}
\setcounter{footnote}{0}
\METAC@ams@maketitle
\begin{quote}
This uses \texttt{maketitle=ams}. Title and author names are in small caps.
Author footnotes are unnumbered (for some reason this is the style
for \texttt{amsart}). Each author's affiliation is listed at the end of the document as below.
\end{quote}

\makeatother
\end{document}
