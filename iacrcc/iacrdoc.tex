% IACR Communications CLASS DOCUMENTATION
% Written by Joppe W. Bos and Kevin S. McCurley (2022),
% based on prior work on the iacrtrans class by
% Gaetan Leurent gaetan.leurent@inria.fr (2016-2018)
%
% To the extent possible under law, the author(s) have dedicated all
% copyright and related and neighboring rights to this software to the
% public domain worldwide. This software is distributed without any
% warranty.
%
% You should have received a copy of the CC0 Public Domain Dedication
% along with this software. If not, see
% <http://creativecommons.org/publicdomain/zero/1.0/>.

\documentclass{iacrcc}

\usepackage{todonotes}

\title[running  = {The iacrcc class},
       onclick  = {https://github.com/IACR/latex},
       subtitle = {LaTeX Class Documentation (v0.10)},
      ]
      {How to Use the IACR Communications in Cryptology Class}

\addauthor[orcid   = {0000-0003-1010-8157},
           inst    = {1},
           onclick = {https://www.joppebos.com},
	   email   = {joppe.bos@nxp.com},
          ]{Joppe W. \surname{Bos}}
\addauthor[orcid   = {0000-0001-7890-5430},
           inst    = {2},
           email   = {mccurley@digicrime.com},
          ]{Kevin S. \surname{McCurley}}

\affiliation[ror      = 031v4g827,
             onclick  = {https://www.nxp.com},
             street   = {Interleuvenlaan 80},
             city     = {Leuven},
             postcode = {3001},
             country  = {Belgium}
            ]{NXP Semiconductors}
\affiliation{Self}

\begin{document}

\maketitle

\keywords{Template \and LaTeX \and IACR}

\begin{abstract}
  This document is a quick introduction to the \LaTeX{} class for the
  \publname{}.
\end{abstract}

\tableofcontents{}

\section*{Introduction}

The \texttt{iacrcc} \LaTeX{} class and \texttt{iacrcc.bst} bibligraphy
style are used by the new IACR Journal titled ``IACR Communications in
Cryptology''.  The biggest difference between this class and other
classes you may have used is the way in which metadata is supplied
(e.g., author information). One goal is to minimize the amount of
human effort required to process final versions of a paper, so that
we can reduce the cost of open access publishing.  Some of
the stylistic choices were inspired by IACR Transactions class file (the
\texttt{iacrtans} class v. 0.92) written by Ga{\"e}tan Leurent and
Friedrich Wiemer.

The class is still in development and feedback and comments are welcome.
The latest version can be found on the Github page of the project:
\url{https://github.com/XXX}, feel free to open
issues or to submit pull requests.
There is a \texttt{template.tex} file included with this class, to use
as a starting point for writing a paper.  The \LaTeX{} source of this
documentation can also be used as a more advanced example, but please
make sure to remove any unnecessary code.

\textbf{NOTE:} one goal of the \texttt{iacrcc.cls} and
\texttt{iacrcc.bst} files is to produce machine-readable metadata in
separate files. This is used in the production steps of the journal,
but should not cause any problems for the author.  Users of
\LaTeX\ will already be used to seeing this with the \texttt{.log},
\texttt{.aux}, \texttt{.bbl}, \texttt{.blg}, \texttt{.toc}, and \texttt{.out} files
produced by \texttt{pdflatex}, \texttt{bibtex}, and use of the
\texttt{hyperref} package.  If your main \LaTeX\ file is called
\texttt{main.tex}, then the extra files that are produced are:
\begin{itemize}
\item \texttt{main.meta} is a flat text file containing all metadata from the paper.
  When you run \texttt{pdflatex} on \texttt{main.tex}, it will produce the metadata
  from \texttt{main.tex}, and when you run \texttt{bibtex} and \texttt{pdflatex} again,
  it will append the citation data from bibtex into the \texttt{main.meta} file as well.
\item \texttt{main.abstract} is a file that contains the contents of the abstract
  for the paper.
\end{itemize}

\section{Invocation and usage}

The class supports four publication types, selected with the
following class options:
\begin{description}
\item[\texttt{[version=final]}] for final papers
\item[\texttt{[version=preprint]}] for preprints (without copyright info, default)
\item[\texttt{[version=submission]}] for submissions (anonymous, with line numbers)
\end{description}
Some types supports further options:
\begin{description}
\item[\texttt{[notanonymous]}] can be used in submission mode, when the
  call for paper requires non-anonymous submissions.
\end{description}

\subsection{Other Options}

\paragraph{\texttt{[xcolor=$\langle\text{\emph{list of options}}\rangle$]}}
\todo[noline]{K: consider removing xcolor. We may want it}
passes $\langle\text{\emph{list of options}}\rangle$ to the
\texttt{xcolor} package.  Since \texttt{xcolor} is loaded by the class,
you have to use this mechanism to pass options at load time; for instance
use \texttt{xcolor=svgnames} to load SVG color names.

\paragraph{\texttt{[hyperref=$\langle\text{\emph{list of options}}\rangle$]}}
passes $\langle\text{\emph{list of options}}\rangle$ to the
\texttt{hyperref} package.  Alternatively, you can load
\texttt{hyperref} yourself with the required options and the class will
detect that it already loaded.

\paragraph{\texttt{[nohyperref]}}
disables the automatic loading of
\texttt{hyperref}.  Use this is if your document fails to compile with
\texttt{hyperref} for some reason.

The \texttt{iacrcc} class automatically loads \texttt{hyperref}
after all other packages.  If you need some packages to be loaded
\emph{after} \texttt{hyperref}, you should load \texttt{hyperref}
explicitly at the correct position, but not use the \texttt{[nohyperref]} option.

\paragraph{\texttt{[nolastpage]}}
disables the automatic loading the \texttt{lastpage} package in
\texttt{[final]} mode.  When this option is enabled, the last page
number must be set explicitly with
\texttt{\textbackslash{}setlastpage}.

Before submitting your final version, please make sure that it compiles
properly using \texttt{lualatex} with the \texttt{[version=final]} option
and check that the author names and affiliation are
correct.  In particular, each affiliation MUST be of the form
$\langle\text{institute}\rangle, \langle\text{city}\rangle,
\langle\text{country}\rangle$.

\paragraph{\texttt{[xcolor=$\langle\text{\emph{list of options}}\rangle$]}}
passes $\langle\text{\emph{list of options}}\rangle$ to the
\texttt{xcolor} package.  Since \texttt{xcolor} is loaded by the class,
you have to use this mechanism to pass options at load time; for instance
use \texttt{xcolor=svgnames} to load SVG color names.

\paragraph{\texttt{[hyperref=$\langle\text{\emph{list of options}}\rangle$]}}
passes $\langle\text{\emph{list of options}}\rangle$ to the
\texttt{hyperref} package.  Alternatively, you can load
\texttt{hyperref} yourself with the required options and the class will
detect that it already loaded.

The \texttt{iacrcc} class automatically loads \texttt{hyperref}
after all other packages.  If you need some packages to be loaded
\emph{after} \texttt{hyperref}, you should load \texttt{hyperref}
explicitly at the correct position.

\paragraph{\texttt{[biblatex]}} uses the biblatex package instead of bibtex.

\subsection{Macros to add title and author information}

The following commands are used to input informations for the title page.

\paragraph{\texttt{\textbackslash{}genericfootnote\{footnotetext\}}}
To place a footnote on the first page without a reference (e.g., to indicate this is a full / extended version of a published paper).
This is an optional macro.

\paragraph{\texttt{\textbackslash title[option=\{xxx\}]\{title\}}} to define the title.
There are a number \emph{optional} options to {\tt \textbackslash{}title}:

\begin{tabular}{l@{\hspace{1cm}}p{0.7\linewidth}}
{\tt running} & The running title displayed in the headers.\\
{\tt onclick} & Define what to do when clicking on the title. This could be used to point to a webpage associated with this paper. \\
{\tt subtitle} & Provide a subtitle.\\
\end{tabular}

\noindent An example using all the optional options would look like:

\begin{verbatim}
\title[running  = {IACR Communications in Cryptology Class},
       onclick  = {https://github.com/IACR/latex},
       subtitle = {A Template}
      ]{How to Use the IACR Communications in Cryptology Class}
\end{verbatim}

\paragraph{\texttt{\textbackslash addauthor[option=\{xxx\}]\{authorname\}}} to define the author list.

Authors are listed individually using the {\tt \textbackslash{}author} command. 
There are a number \emph{optional} options to {\tt \textbackslash{}addauthor}:

\begin{tabular}{l@{\hspace{1cm}}p{0.7\linewidth}}
{\tt inst} & A numerical list pointing to the index of the institutaion in the affiliation array.\\
{\tt orcid} & Create a small clickable orcid logo next to the authors name looking like \orcidlink{0000-0003-1010-8157} and linking to the authors ORCID iD (see: \url{https://orcid.org}.\\
{\tt footnote} & Create an author-specific footnote.\\
{\tt onclick} & Define what to do when clicking on the author name: e.g.,~can point to the academic webpage.\\
{\tt email} & Define the e-mail address of this author.\\
\end{tabular}

\noindent Moreover, one can utilize the {\tt \textbackslash{}surname} macro to indicate what part of the author name is the surname:
this is used for meta-data collection and is especially useful for cases with middle names \emph{and} double 
surnames which might be confusing. 

An example using all the optional options would look like:

\begin{verbatim}
\author[orcid    = 0000-0000-0000-0000,
        inst     = {1,2},
        footnote = {Thanks to my supervisor for the support.},
        onclick  = {https://www.mypersonalwebpage.com},
        email    = {alice@accomplished.com},
       ]{Alice \surname{Accomplished}}
\end{verbatim}

\paragraph{\texttt{\textbackslash{}affiliations[option=\{xxx\}]\{affiliation\}}}
Affiliations are listed individually using the {\tt \textbackslash{}affiliations} command \emph{after}
the (list of) authors using {\tt \textbackslash{}addauthor}.
There are a number \emph{optional} options to {\tt \textbackslash{}affiliation}:

\begin{tabular}{l@{\hspace{1cm}}p{0.7\linewidth}}
{\tt ror} & Provide the Research Organization Registry (ROR) indetifier for this affiliation (see: \url{https://ror.org}). This is used for meta-data collection only.\\
{\tt onclick} & Define what to do when clicking on the affiliation name: e.g.,~can point to the affiliation webpage.\\
{\tt  department} & Department or suborganization name.\\
{\tt  street} & Street address.\\
{\tt  city} & City name.\\
{\tt  state} & State or province name.\\
{\tt  postcode} & zip or postal code.\\
{\tt  country} & Country name.\\
\end{tabular}

\noindent An example using all the optional options would look like:

\begin{verbatim}
\affiliation[ror        = 05f950310,
             onclick    = {http://www.kuleuven.be/english},
             department = {Computer Security and Industrial Cryptography},              
             street     = {Kasteelpark Arenberg 10, box 2452},
             city       = {Leuven},
             state      = {Vlaams-Brabant},
             postcode   = {3001},
             country    = {Belgium}
            ]{KU Leuven}
\end{verbatim}           

\paragraph{\texttt{\textbackslash license\{license\}}}
When the ``final'' document mode is used the author needs to provide a supported license.
In all other modes this information is not required and when provided ignored.
The currently provided set of licences are \mbox{\{ CC-by \}}.
An example would look like:

\begin{verbatim}
\license{CC-by}
\end{verbatim}

\paragraph{\texttt{\textbackslash keywords\{keyword1 \textbackslash{}and keyword2\}}} to give a list of
keywords.
Individual keywords should be separated by the \verb+\and+ macro.
This is an optional macro.

\paragraph{\texttt{\textbackslash maketitle}} is used to actually
typeset the title.

\paragraph{The \texttt{abstract} environment} should be used to typeset the abstract.

Note that the keywords should be given before starting the abstract environment.

\subsection{Theorems}

The \texttt{iacrcc} class uses the \AmS{} packages to typeset
math.  In particular, it loads the \texttt{amsthm} package, and
predefines the following environments:
\begin{center}
  \ttfamily
\begin{tabular}{l@{\hspace{1cm}}l@{\hspace{1cm}}l}
theorem     & definition & remark \\
proposition & example    & note   \\
problem     & exercise   & case   \\
lemma       & property   &        \\
conjecture  & question   &        \\
corollary   & solution   &        \\
claim       &            &        \\
\end{tabular}
\end{center}

Note that the \texttt{proof} environment automatically adds a QED
symbol at the end of the proof.
If the QED symbol
is typeset at a wrong position, you can force its position with
\verb+\qedhere+.

\section{Typesetting the Bibliography}
\label{sec:biblio}

% Borrowed from btxdoc.tex
\def\BibTeX{{\rm B\kern-.05em{\sc i\kern-.025em b}\kern-.08em
    T\kern-.1667em\lower.7ex\hbox{E}\kern-.125emX}}

Having good bibliographic references is very important for the
visibility of the journal.  Since we don't have a commercial editor,
authors need to make sure themselves that references are standardized
and clean.  We strongly encourage authors to use bibliographic data
from \url{http://www.dblp.org} or \url{https://cryptobib.di.ens.fr/}.
All references should have DOIs if at all possible.

The author may use either \BibTeX\ or \texttt{biblatex}. If the author
uses \BibTeX\ then they should use \verb+\bibliographystyle{iacrcc}+.
If they use \texttt{biblatex}, then this is done using
\verb+\documentclass[biblatex]{iacrcc}+ instead of \verb+\usepackage{biblatex}+;
the latter will generate an error because the \texttt{iacrcc.cls} file
loads \texttt{biblatex} with a specific style.

Here are some example citations: the RSA paper~\cite{RSA78}, the AES
standard\cite{AES-FIPS}, and \cite{DBLP:conf/crypto/Kocher96}.

\section{Further instructions}

\paragraph{\LaTeX{} distribution, and worklow.}  \LaTeX{}
distributions are available on a variaty of platforms.  In particular,
we recommand the \href{https://www.tug.org/texlive/}{TeX Live}
distribution, which is updated regularly, include a large number of
packages, and is available on many platforms.
\begin{description}
\item[Linux:] A LaTeX installation is included in most Linux
  distributions.  Alternatively,
  \href{https://www.tug.org/texlive/}{TeX Live} can be installed
  easily without root access.
\item[Windows:] There are also good \LaTeX{} distributions for Windows,
  such as \href{http://www.miktex.org/}{MikTeX} and
  \href{https://www.tug.org/texlive/}{TeX Live}.
\item[MacOSX:] On MacOSX, TeX Live is available inside
  \href{http://www.tug.org/mactex/}{MacTeX}.
\end{description}


We recommand the use of \texttt{pdflatex} because it generally
supports more features than \texttt{latex} and \texttt{dvips}
(\texttt{xelatex} and \texttt{lualatex} are also missing some advanced
features from \texttt{pdflatex}).

\paragraph{Internal references.}

We recommend the use of \verb+\autoref+ from \texttt{hyperref}
(automatically loaded by the class).  For instance,
\verb+\autoref{sec:options}+ links to \autoref{sec:options}.

\paragraph{Pictures.}

We recommend the use of the \texttt{tikz} package to render pictures.

In particular, a large variety of crypto pictures made with
\texttt{tikz} is available at \url{http://www.iacr.org/authors/tikz/}.

\paragraph{External pictures.}  The \texttt{graphicx} is loaded by the
class, and is recommended for external figures.

If possible, external figures should be in a vector format: you can
use PDF files when compiling with \texttt{pdflatex}, and EPS files
when compiling with \texttt{latex}, and \texttt{dvips}.  Note that the
\verb+\includegraphics+ command will automatically select a file with
the right extension, so if you write \verb+\includegraphics{figure}+
and have two files \texttt{figure.pdf} and \texttt{figure.eps}, it
should work with both workflow.

\paragraph{Floats.}

Figure captions should be below the figures, and table captions above
the tables.  The \texttt{float} package loaded by the class should
take care of this automatically.  If want to have several figures side
by side, see the \texttt{[floatrow]} option.

\paragraph{Tables.}

We recommend the \texttt{booktabs} package to typeset tables.

\paragraph{Algorithms.}

We recommend the \texttt{algorithm}, \texttt{algorithmcx} packages for
algorithms (in particular, \texttt{algpseudocode} for pseudo-code).


\section{For the Editor}

The following commands should be used by the editor to prepare the final
version:

\begin{itemize}
\item \texttt{\textbackslash{}setfirstpage} to set the first page number.
\item \texttt{\textbackslash{}setlastpage} to set the last page number (optional).
\item \texttt{\textbackslash{}setvolume} to set the volume number.
\item \texttt{\textbackslash{}setnumber} to set the edition number.
\item \texttt{\textbackslash{}setDOI} to set the DOI\@.
\item \texttt{\textbackslash{}setPublished} to set the publication date.
\item \texttt{\textbackslash{}setAccepted} to set the notification date.
\item \texttt{\textbackslash{}setRevised} to set the re-submission date for paper that went though major revision.
\item \texttt{\textbackslash{}setReceived} to set the submission date.
\end{itemize}

There is a special \texttt{settings.tosc.tex} file, that sets default values for these commands
and which can be included in the beginning of the main tex file.

\section{Further information}

More general information can be found in the following documents:
\begin{itemize}
\item General \LaTeX{} documentation, such as the
  \href{http://mirrors.ctan.org/info/lshort/english/lshort.pdf}{(not
    so) short introduction to \LaTeXe};
% \item The \texttt{article} class
%   \href{http://mirrors.ctan.org/macros/latex/doc/clsguide.pdf}{documentation};
\item The \AmS-\LaTeX{}
  \href{http://mirrors.ctan.org/macros/latex/required/amslatex/math/amsldoc.pdf}{documentation}
  and \texttt{amsthm} \href{ftp://ftp.ams.org/pub/tex/doc/amscls/amsthdoc.pdf}{documentation};
%\item Documentation of the \LaTeX{} packages used in the class (see below).
\end{itemize}

\section*{Thanks}
% Keep this old Thanks list? If so extend!
We would like to thank people who helped design and improve the class:
Anne Canteaut,
Jérémy Jean,
Marc Joye,
Bart Preneel,
Christian Rechberger,
Tyge Tiessen,
Jonas Wloka.

\renewcommand{\refname}{Sample References}
\bibliography{biblio}
\end{document}
