% IACR Communications CLASS DOCUMENTATION
% Written by Joppe W. Bos and Kevin S. McCurley (2022),
% based on prior work on the iacrtrans class by
% Gaetan Leurent gaetan.leurent@inria.fr (2016-2018)
%
% To the extent possible under law, the author(s) have dedicated all
% copyright and related and neighboring rights to this software to the
% public domain worldwide. This software is distributed without any
% warranty.
%
% You should have received a copy of the CC0 Public Domain Dedication
% along with this software. If not, see
% <http://creativecommons.org/publicdomain/zero/1.0/>.

\documentclass{iacrcc}
\usepackage{todonotes}

\title[running  = {The iacrcc class},
       onclick  = {https://github.com/IACR/latex},
       subtitle = {LaTeX Class Documentation (v0.10)},
      ]
      {How to Use the IACR Communications in Cryptology Class}

\addauthor[orcid   = {0000-0003-1010-8157},
           inst    = {1},
           onclick = {https://www.joppebos.com},
	   email   = {joppe.bos@nxp.com},
          ]{Joppe W. \surname{Bos}}
\addauthor[orcid   = {0000-0001-7890-5430},
           inst    = {2},
           email   = {iacrdoc@digicrime.com},
          ]{Kevin S. \surname{McCurley}}

\affiliation[ror      = 031v4g827,
             onclick  = {https://www.nxp.com},
             street   = {Interleuvenlaan 80},
             city     = {Leuven},
             postcode = {3001},
             country  = {Belgium}
            ]{NXP Semiconductors}
\affiliation{Self}
\license{CC-by}
\begin{document}

\maketitle

\keywords[Template, LaTeX, IACR]{Template, \LaTeX, IACR}

\begin{abstract}
  This document is a quick introduction to the \texttt{iacrcc.cls}
  \LaTeX{} class for the \publname{}.
\end{abstract}

\tableofcontents{}

\section*{Introduction}

This document is a guide to preparing a paper for the IACR journal
titled ``IACR Communications in Cryptology''.  Papers for the journal
must be prepared in \LaTeX\ using a new \texttt{iacrcc} document
class. The biggest difference between \texttt{iacrcc} and other
classes you may have used is the way in which metadata is supplied
(e.g., author information). Our goal is to minimize the amount of
human effort required to process final versions of a paper, so that we
can reduce the cost of open access publishing.  Some of the stylistic
choices were inspired by IACR Transactions class file (the
\texttt{iacrtrans} class v. 0.92) written by Ga{\"e}tan Leurent and
Friedrich Wiemer with help from others.

The class is still in development and feedback and comments are
welcome.  The latest version can be found at:
\href{https://publish.iacr.org/iacrcc}{\texttt{publish.iacr.org/iacrcc}}.
There is also a \href{https://github.com/IACR/latex}{github
  repository} where you can open issues or to submit pull requests.
The complete package consists of four files. \texttt{iacrcc.cls} is
the main document class. \texttt{iacrcc.bst} is the bibliography style
for the journal.  \texttt{iacrdoc.tex} is used to produce this PDF
file. \texttt{template.tex} can be used as a starting point for
writing a paper.  The \LaTeX{} source of \texttt{iacrdoc.tex} can also
be used as a more advanced example, but please make sure to remove any
unnecessary code.

\paragraph*{NOTES}
\begin{itemize}
\item The production system to which you submit your final version requires
  that the main \LaTeX\ file should be \texttt{main.tex}. You might as
  well start that way.
\item The production versions of papers produced with this style will
  be processed with \texttt{lualatex}.  You can use \texttt{pdflatex}
  for development, but it lacks modern features for support of UTF-8
  text and we recommend using \texttt{lualatex} instead.  Do not use
  any features that are specific to \texttt{pdflatex}, because they
  will not work in final versions. In particular, do not use the
  \texttt{fontenc} package; use \texttt{fontspec} instead.
\item One goal of the \texttt{iacrcc.cls} and \texttt{iacrcc.bst}
  files is to automate the production of machine-readable metadata in
  separate files. Users of \LaTeX\ will already be used to seeing this
  with the \texttt{.log}, \texttt{.aux}, \texttt{.bbl}, \texttt{.blg},
  \texttt{.toc}, and \texttt{.out} files produced by \texttt{bibtex}
  and the \texttt{hyperref} package.  If your main \LaTeX\ file is
  called \texttt{main.tex}, then the extra files that are produced
  are:
\begin{itemize}
\item \texttt{main.meta} is a flat text file containing all metadata
  from the paper.  When you run \texttt{lualatex} on
  \texttt{main.tex}, it will produce the metadata from
  \texttt{main.tex}, and when you run \texttt{bibtex} and
  \texttt{lualatex} again, it will append the citation data from
  bibtex into the \texttt{main.meta} file as well.
\item \texttt{main.abstract} is a file that contains the contents of
  the abstract for the paper.
\end{itemize}
\end{itemize}
\section{Invocation and usage}

The class supports the following class options with \verb+\documentclass{iacrcc}+
\begin{description}
\item[\texttt{[version=preprint]}] for preprints (without copyright info, default)
\item[\texttt{[version=submission]}] for submissions (anonymous, with
  line numbers).  if desired, this can be combined with
  \texttt{[notanonymous]} if the call for papers requires
  non-anonymous submissions.
\item[\texttt{[version=final]}] for final papers. This must be
  compiled with \texttt{lualatex} and imposes some additional
  requirements.
\item[\texttt{[biblatex]}] may be used if you prefer using
  \texttt{biblatex} to \texttt{bibtex}. Note that we do not support options to be
  passed to biblatex, as they may conflict with the style of the journal.
\item[\texttt{[floatrow]}] load the \texttt{floatrow} package. This is useful
  when you have fancy figures or tables.
\end{description}

Before submitting your final version, please make sure that it compiles
properly using \texttt{lualatex} with the \texttt{[version=final]} option
and check that the author names and affiliation are
correct.

\subsection{Other Options}
There are a few other options to control how packages work, but authors should never
need these. Use with caution.

\paragraph{\texttt{[xcolor=$\langle\text{\emph{list of options}}\rangle$]}}
\todo[inline]{K: I think we should remove this. For one thing, it can
  interfere with PDF/A. Second, authors can screw up things like the
  backend color driver being specified as pdftex. I looked at tosc and tches papers
  from 2020-22, and found no real use of this in iacrtrans. I just don't think
  we need it, so we should let someone ask for it and give a good
  reason.}  passes $\langle\text{\emph{list of options}}\rangle$ to
the \texttt{xcolor} package.  Since \texttt{xcolor} is loaded by the
class, you have to use this mechanism to pass options at load time;
for instance use \texttt{xcolor=svgnames} to load SVG color names. Do
not use this unless you absolutely know what you are doing.

\paragraph{\texttt{[hyperref=$\langle\text{\emph{list of options}}\rangle$]}}
\todo[inline]{K: I also think we should remove the hyperref
  options. It gives authors a lot of opportunity to screw things up. I
  looked at tches and tosc papers from 2020-2022 and found only a
  couple of people who used this:
  \texttt{[backref=true]},\texttt{[hidelinks=true]}, and
  \texttt{[pagebackref=false,bookmarks=false]}. None of these make sense for us, and I don't
  want everyone to have different colors for things.}  passes
$\langle\text{\emph{list of options}}\rangle$ to the \texttt{hyperref}
package.  Alternatively, you can load \texttt{hyperref} yourself with
the required options and the class will detect that it already loaded.

The \texttt{iacrcc} class automatically loads \texttt{hyperref}
after all other packages.  If you need some packages to be loaded
\emph{after} \texttt{hyperref}, you should load \texttt{hyperref}
explicitly at the correct position, but not use the \texttt{[nohyperref]} option.

\subsection{Macros to add title and author information}

A title is added using the \verb+\title+ macro, but with a few other options:
\begin{center}
\begin{tabular}{l@{\hspace{1cm}}p{0.7\linewidth}}
{\tt running} & The running title displayed in the headers.\\
{\tt onclick} & Define what to do when clicking on the title. This could be used to point to a webpage associated with this paper. \\
{\tt subtitle} & Provide a subtitle.\\
\end{tabular}
\end{center}
\noindent  Do not use author-defined macros in arguments to
\texttt{\textbackslash title}. An example using all the optional
options would look like:

\begin{verbatim}
\title[running  = {IACR Communications in Cryptology Class},
       onclick  = {https://github.com/IACR/latex},
       subtitle = {A Template}
      ]{How to Use the IACR Communications in Cryptology Class}
\end{verbatim}

Author information is entered using the \verb+\addauthor+,
\verb+\affiliation+, and \verb+\addfunding+ macros. Authors are asked
to enter this information in a structured way so that we can provide
it to indexing agencies. The \verb+\author+ macro is disabled.

\paragraph{\texttt{\textbackslash addauthor[option=\{xxx\}]\{authorname\}}}
Authors are listed individually using the {\tt \textbackslash{}addauthor} command. 
There are a number of \emph{optional} options to {\tt \textbackslash{}addauthor}:
\begin{center}
  \begin{tabular}{l@{\hspace{1cm}}p{0.7\linewidth}}
    {\tt inst} & A numerical list of indices specifying an institution in the affiliation
    array (see below).\\
{\tt orcid} & Create a small clickable orcid logo next to the authors name looking like \orcidlink{0000-0003-1010-8157} and linking to the authors ORCID iD (see: \url{https://orcid.org}.\\
{\tt footnote} & Create an author-specific footnote.\\
{\tt onclick} & provide a URL to visit when clicking on the author name: e.g.,~can
point to the academic webpage.\\
{\tt email} & Define the e-mail address of this author.\\
\end{tabular}
\end{center}

\noindent We \textbf{strongly} recommend that authors enter their
ORCID ID into the paper, because this ensures that they will get
citation credit for their papers. Authors can use the {\tt
  \textbackslash{}surname} macro to indicate what part of the author
name is the surname: this is used for meta-data collection and is
especially useful for cases with middle names \emph{and} double
surnames which might be confusing.

An example using all the optional options is given below. In this case
the author has \verb+inst={1,2}+ to indicate that they are affiliated with
the the first and second affiliations that are entered with
\verb+\affiliation+

\begin{verbatim}
\addauthor[orcid    = 0000-0000-0000-0000,
        inst     = {1,2},
        footnote = {Thanks to my supervisor for the support.},
        onclick  = {https://www.mypersonalwebpage.com},
        email    = {alice@accomplished.com},
       ]{Alice \surname{Accomplished}}
\end{verbatim}

\paragraph{\texttt{\textbackslash{}affiliation[option=\{xxx\}]\{affiliation\}}}
Affiliations are listed individually using the {\tt
  \textbackslash{}affiliation} command \emph{after} the last author
has been added using {\tt \textbackslash{}addauthor}.  There are a number
of arguments to {\tt \textbackslash{}affiliation}:

%\begin{tabular}{l@{\hspace{1cm}}p{0.7\linewidth}}
\begin{center}
  \begin{tabular}{l@{\hspace{1cm}}p{0.7\linewidth}}
    {\tt ror} & Provide the Research Organization Registry (ROR) indentifier
    for this affiliation (see: \url{https://ror.org}). This is used for meta-data collection only.\\
{\tt onclick} & Define what to do when clicking on the affiliation name: e.g.,~can point to the affiliation webpage.\\
{\tt  department} & Department or suborganization name.\\
{\tt  street} & Street address.\\
{\tt  city} & City name.\\
{\tt  state} & State or province name.\\
{\tt  postcode} & zip or postal code.\\
     {\tt  country} & Country name. Required for \texttt{[version=final]}.\\
\end{tabular}
\end{center}
\noindent There is an online tool at
\href{https://publish.iacr.org/funding}{\texttt{publish.iacr.org/funding}} to
help you find ROR identifiers.  An example using all the optional
options would look like:

\begin{verbatim}
\affiliation[ror        = 05f950310,
             onclick    = {http://www.kuleuven.be/english},
             department = {Computer Security and Industrial Cryptography},              
             street     = {Kasteelpark Arenberg 10, box 2452},
             city       = {Leuven},
             state      = {Vlaams-Brabant},
             postcode   = {3001},
             country    = {Belgium}
            ]{KU Leuven}
\end{verbatim}           

\paragraph{\texttt{\textbackslash addfunding[option={xxx}]\{agency name\}}}
Authors should list funding information in order to make sure that funding agencies
can find papers published under their sponsorship. An example is:
\begin{verbatim}
\addfunding[fundref = 100000001
            grantid = {CNS-1237235},
            country = {United States}]{National Science Foundation}
\addfunding[ror=00pn5a327,
            country={United States}]{Rambus}
\end{verbatim}

\noindent In this example, the author acknowledges a grant from the
National Science Foundation and support from Rambus (with no
\texttt{grantid}). The inclusion of funding from an agency without a
\texttt{grantid} might be appropriate if the author simply received
support for a visit.

The complete list of options for \texttt{\textbackslash addfunding} is:
\begin{center}
\begin{tabular}{l@{\hspace{1cm}}p{0.7\linewidth}}
  {\tt fundref} & An identifier from the
  \href{https://publish.iacr.org/funding}{Crossref funder registry}\\
  {\tt ror} & An identifier from the \href{https://publish.iacr.org/funding}{Research Organization Registry} (ROR). A \texttt{fundref} identifier is preferred for \texttt{\textbackslash addfunding}.\\
  {\tt country} & The country of the funding agency \\
  {\tt grantid} & The identifier of the grant that is assigned by the agency who
      provided it
\end{tabular}
\end{center}
\noindent You can use the online tool at 
\href{https://publish.iacr.org/funding}{\texttt{publish.iacr.org/funding}} to
help you find \texttt{fundref} and \texttt{ror} identifiers.

\paragraph{\texttt{\textbackslash license\{license\}}}
When the \texttt{version=final} document mode is used, the author needs
to provide a supported license.  In all other modes this information
is not required and when provided ignored.  At present the only
acceptable license is \texttt{CC-by}.  An example would look like:

\begin{verbatim}
\license{CC-by}
\end{verbatim}

\paragraph{\texttt{\textbackslash keywords\{keyword1, keyword2\}}} to give a list of
keywords. This is an optional macro.  Individual keywords should be
separated by commas. If the argument to \texttt{\textbackslash keywords}
contains math or macros, then you must supply an additional set of
text-only keywords; for example:


\texttt{\textbackslash keywords[rings, arithmetic on Z]\{rings, arithmetic on \$\textbackslash mathbb\{Z\}\$\}}

\paragraph{\texttt{\textbackslash maketitle}} is used to actually
typeset the title.

\paragraph{\texttt{\textbackslash{}genericfootnote\{footnotetext\}}}
To place a footnote on the first page without a reference (e.g., to
indicate this is a full / extended version of a published paper).
This is an optional macro.

\paragraph{The \texttt{abstract} environment} Abstracts are entered with
the \texttt{abstract} environment as usual, but do not use
author-defined macros within the abstract.  Note that the keywords
should be given before starting the abstract environment.

\subsection{Theorems}

The \texttt{iacrcc} class uses the \AmS{} packages to typeset
math.  In particular, it loads the \texttt{amsthm} package, and
predefines the following environments:
\begin{center}
  \ttfamily
\begin{tabular}{l@{\hspace{1cm}}l@{\hspace{1cm}}l}
theorem     & definition & remark \\
proposition & example    & note   \\
problem     & exercise   & case   \\
lemma       & property   &        \\
conjecture  & question   &        \\
corollary   & solution   &        \\
claim       &            &        \\
\end{tabular}
\end{center}

Note that the \texttt{proof} environment automatically adds a QED
symbol at the end of the proof.
If the QED symbol
is typeset at a wrong position, you can force its position with
\verb+\qedhere+.

\section{Typesetting the Bibliography}
\label{sec:biblio}

% Borrowed from btxdoc.tex
\def\BibTeX{{\rm B\kern-.05em{\sc i\kern-.025em b}\kern-.08em
    T\kern-.1667em\lower.7ex\hbox{E}\kern-.125emX}}

Having good bibliographic references is very important for the
visibility of the journal.  Since we don't use a commercial editor,
authors need to make sure themselves that references are standardized
and clean.  We strongly encourage authors to use bibliographic data
from \url{http://www.dblp.org} or \url{https://cryptobib.di.ens.fr/}.
All references should have DOIs if at all possible.

The author may use either \BibTeX\ or \texttt{biblatex}. If the author
uses \BibTeX, then the \texttt{iacrcc} class will load the
\verb+\bibliographystyle{iacrcc}+ style.  If they use
\texttt{biblatex}, then this is done using
\verb+\documentclass[biblatex]{iacrcc}+ instead of
\verb+\usepackage{biblatex}+; the latter will generate an error
because the \texttt{iacrcc.cls} file loads \texttt{biblatex} with a
specific style.

Here are some example citations: the RSA paper~\cite{RSA78}, the AES
standard\cite{AES-FIPS}, and \cite{DBLP:conf/crypto/Kocher96}.

\section{Some recommendations}\label{sec:options}

\paragraph{\LaTeX{} distribution, and worklow.}  \LaTeX{}
distributions are available on a variaty of platforms.  In particular,
we recommend the \href{https://www.tug.org/texlive/}{TeX Live}
distribution, which is updated regularly, include a large number of
packages, and is available on many platforms. We use this for our
cloud service to compile final versions of papers.
\begin{description}
\item[Linux:] A LaTeX installation is included in most Linux
  distributions.  Alternatively,
  \href{https://www.tug.org/texlive/}{TeX Live} can be installed
  easily without root access.
\item[Windows:] There are also good \LaTeX{} distributions for Windows,
  such as \href{http://www.miktex.org/}{MikTeX} and
  \href{https://www.tug.org/texlive/}{TeX Live}.
\item[MacOSX:] On MacOSX, TeX Live is available inside
  \href{http://www.tug.org/mactex/}{MacTeX}.
\end{description}

\paragraph{Internal references.}

We recommend the use of \verb+\autoref+ from \texttt{hyperref}
(automatically loaded by the class).  For instance,
\verb+\autoref{sec:options}+ links to \autoref{sec:options}.

\paragraph{Pictures.}

We recommend the use of the \texttt{tikz} package to render pictures.
In particular, a large variety of crypto pictures made with
\texttt{tikz} is available at \href{http://iacr.org/authors/tikz/}{\texttt{iacr.org/authors/tikz/}}

\paragraph{External pictures.}  The \texttt{graphicx} is loaded by the
class, and is recommended for external figures. The submission server
does not support \texttt{svg} format for included graphics, so
you should convert \texttt{svg} files to a supported format.

If possible, external figures should be in a vector format (PDF or
EPS).  Note that the \verb+\includegraphics+ command will
automatically select a file with the right extension, so if you write
\verb+\includegraphics{figure}+ and have two files \texttt{figure.pdf}
and \texttt{figure.eps}, it should work with both workflow.

\paragraph{Floats.}

Figure captions should be below the figures, and table captions above
the tables.  The \texttt{float} package loaded by the class should
take care of this automatically.  If want to have several figures side
by side, see the \texttt{[floatrow]} option.

\paragraph{Tables.}

We recommend the \texttt{booktabs} package to typeset tables.

\paragraph{Algorithms.}

We recommend the \texttt{algorithmicx} packages for algorithms (in
particular, \texttt{algpseudocodex} for pseudo-code).


\section{Further information}

More general information can be found in the following documents:
\begin{itemize}
\item General \LaTeX{} documentation, such as the
  \href{http://mirrors.ctan.org/info/lshort/english/lshort.pdf}{(not
    so) short introduction to \LaTeXe};
% \item The \texttt{article} class
%   \href{http://mirrors.ctan.org/macros/latex/doc/clsguide.pdf}{documentation};
\item The \AmS-\LaTeX{}
  \href{http://mirrors.ctan.org/macros/latex/required/amslatex/math/amsldoc.pdf}{documentation}
  and \texttt{amsthm} \href{ftp://ftp.ams.org/pub/tex/doc/amscls/amsthdoc.pdf}{documentation};
%\item Documentation of the \LaTeX{} packages used in the class (see below).
\end{itemize}

\renewcommand{\refname}{Sample References}
\bibliography{biblio}
\end{document}
