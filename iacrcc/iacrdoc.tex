% IACR Communications CLASS DOCUMENTATION
% Written by Joppe W. Bos and Kevin S. McCurley (2022-2024),
% based on prior work on the iacrtrans class by
% Gaetan Leurent gaetan.leurent@inria.fr (2016-2018).
%
% To the extent possible under law, the author(s) have dedicated all
% copyright and related and neighboring rights to this software to the
% public domain worldwide. This software is distributed without any
% warranty.
%
% You should have received a copy of the CC0 Public Domain Dedication
% along with this software. If not, see
% <http://creativecommons.org/publicdomain/zero/1.0/>.

\DocumentMetadata{pdfstandard=a-2u,pdfversion=1.7}
\documentclass{iacrcc}
\newcommand{\cmd}[2][]{%
  \def\FirstArg{#1}
  \ifx\FirstArg\empty
    \texttt{\textbackslash{}#2}%
  \else
    \texttt{\textbackslash{}#2\{#1\}}%
  \fi
}

% Borrowed from btxdoc.tex
\def\BibTeX{{\rm B\kern-.05em{\sc i\kern-.025em b}\kern-.08em
    T\kern-.1667em\lower.7ex\hbox{E}\kern-.125emX}}

\def\BibLaTeX{Bib\LaTeX}

\title[running  = {The iacrcc class},
       subtitle = {iacrcc LaTeX Class Documentation (v0.75)}
      ]{How to Use the IACR Communications in Cryptology Class}

\genericfootnote{This is a generic footnote produced with \cmd[...]{genericfootnote}.}

\addauthor[orcid   = {0000-0003-1010-8157},
           inst    = {1},
           onclick = {https://www.joppebos.com},
	   email   = {joppe.bos@nxp.com},
	   surname = {Bos},
          ]{Joppe W. Bos}
\addauthor[orcid   = {0000-0001-7890-5430},
           inst    = {2},
           email   = {iacrdoc@digicrime.com},
	   surname = {McCurley},
          ]{Kevin S. McCurley}

\addaffiliation[ror      = {031v4g827},
                street   = {Interleuvenlaan 80},
                city     = {Leuven},
                postcode = {3001},
                country  = {Belgium}
               ]{NXP Semiconductors}
\addaffiliation[country={United States}]{Self}

\license{CC-by}

\begin{document}

\maketitle

\keywords[Template, LaTeX, IACR]{Template, \LaTeX, IACR}

\begin{abstract}
  This document is a quick introduction to the \texttt{iacrcc.cls}
  \LaTeX{} class for the \publname{}.
\end{abstract}

\begin{textabstract}
  This document is a quick introduction to the iacrcc.cls
  LaTex class for the IACR Communications in Cryptology.
\end{textabstract}

\section*{Introduction}

This document is a guide to preparing a paper for the IACR journal
titled ``IACR Communications in Cryptology''.  Papers for the journal
must be prepared in \LaTeX\ using the \texttt{iacrcc} document
class. The biggest difference between \texttt{iacrcc} and other
classes you may have used is the way in which metadata is supplied
(e.g., title, author information, affiliations, and funding
information). Our goal is to minimize the amount of human effort
required to process final versions of a paper, so that we can reduce
the cost of open access publishing.  Some of the stylistic choices
were inspired by IACR Transactions class file (the \texttt{iacrtrans}
class v. 0.92) written by Ga{\"e}tan Leurent and Friedrich Wiemer with
help from others.

The class is still in development and feedback and comments are
welcome.  The latest version can be found at:
\href{https://publish.iacr.org/iacrcc}{\texttt{publish.iacr.org/iacrcc}}.
There is also a \href{https://github.com/IACR/latex}{github
  repository} where you can open issues or to submit pull requests.
The complete package consists of three files. \texttt{iacrcc.cls} is
the main document class. \texttt{iacrdoc.tex} is used to produce this PDF
file. \texttt{template.tex} can be used as a starting point for
writing a paper.  The \LaTeX{} source of \texttt{iacrdoc.tex} can also
be used as a more advanced example, but please make sure to remove any
unnecessary code.

\paragraph*{NOTES}
\begin{itemize}
\item The production system to which you submit your final version requires
  that the main \LaTeX\ file should be \texttt{main.tex}. You might as
  well start that way by copying \texttt{template.tex} to \texttt{main.tex}.
\item The default fonts are provided by the \verb+lmodern+ package. Do not
  change this.
\item Avoid using too many packages. Many authors are lazy and just
  copy what they used in the past. Some won't work - see the list of
  acceptable packages at \url{https://publish.iacr.org}. Don't use any
  package that changes fonts.
\item Don't try to change the \texttt{hyperref} options, the bibliography style,
  the page style, or the page numbering.
\item Don't use macros like \cmd{if} or \cmd{include} inside any
  metadata like the title or abstract.
\item Footnotes are handled differently in this class, and in
  particular \cmd{thanks} is disabled.
\end{itemize}

\section{Invocation and usage}

The class supports the following class options with \verb+\documentclass{iacrcc}+
\begin{description}
\item[\texttt{[version=preprint]}] for preprints (without copyright info, default)
\item[\texttt{[version=submission]}] for submissions (anonymous, with
  line numbers).  If desired, this can be combined with
  \texttt{[notanonymous]} if the call for papers requires
  non-anonymous submissions.
\item[\texttt{[version=final]}] for final papers. This imposes some
  additional requirements.
\item[\texttt{[biblatex]}] may be used if you prefer using
  \BibLaTeX\ to \BibTeX. Note that we do not support options to be
  passed to \BibLaTeX, as they may conflict with the style of the journal. We use
  a bibliography style based on the alpha style (e.g., \cite{RSA78}).
\item[\texttt{[floatrow]}] load the \texttt{floatrow} package. This is useful
  when you have fancy figures or tables. In either case, \texttt{iacrcc}
  will customize how tables and figures are laid out.
\end{description}
An example how to pass an option to this document class is
\sloppy\verb+\documentclass[version=submission]{iacrcc}+
for submissions.

Before submitting your final version, please make sure that it compiles
properly with the \texttt{[version=final]} option
and check that the author names and affiliations are
correct and a text version of the abstract is provided
in the \texttt{textabstract} environment.

The current date of compilation time is automatically added to the
footer of the front page. If you want to adjust this date you can
use the \verb+\documentdate+ macro (e.g. \verb+\documentdate{2023-10-05}+)
or use \verb+\documentdate{}+ to omit adding the date.

The \texttt{iacrcc} class automatically loads \texttt{hyperref}
after all other packages.  If you need some packages to be loaded
\emph{after} \texttt{hyperref}, you should read Section~\ref{sec:loadorder}.

\section{Macros to add title and author information}

\subsection{Title}
A title is added using the \cmd{title} macro, it has a number of optional arguments:
\begin{center}
  \begin{tabular}{l@{\hspace{1cm}}p{0.7\linewidth}}
    {\tt running}   & The running title displayed in the headers.\\
    {\tt plaintext} & A text version of the title (mandatory if macros are used in the title).\\
    {\tt subtitle}  & Provide a subtitle.\\
  \end{tabular}
\end{center}
\noindent An example using all the optional arguments would look like:

\begin{verbatim}
\title[running   = {How to use the iacrcc class},
       plaintext = {How to use the iacrcc LaTeX class},
       subtitle  = {A Template},
      ]{How to use the \texttt{iacrcc} \LaTeX\ class}
\end{verbatim}
The \verb+plaintext+ option is only required if you use macros in
your title (it is required in the example). Inline mathematics and
accents like \verb+\"u+ are allowed in plain text. Note
that \LaTeX\ has defaulted to UTF-8 input since 2019, so just ü is
preferred to \verb+\"u+. Note that \verb+\thanks+ and \verb+\footnote+ are disabled
and you should instead use the
\cmd{genericfootnote} macro described in section~\ref{footnotes}.

\subsection{Authors}
Author information is entered using the \verb+\addauthor+,
\verb+\addaffiliation+, and \verb+\addfunding+ macros. Authors are asked
to enter this information in a structured way so that we can provide
it to indexing agencies. The \verb+\author+ macro is disabled.

Authors are listed individually using repeated calls to the {\tt
\textbackslash{}addauthor} command.  There are a number of
optional arguments to {\tt \textbackslash{}addauthor}:
\begin{center}
  \begin{tabular}{l@{\hspace{1cm}}p{0.7\linewidth}}
    {\tt inst}     & A numerical list of indices specifying an institution in the 
                     affiliation array (see below).\\
    {\tt orcid}    & Create a small clickable orcid logo next to the authors name 
                     looking like \orcidlink{0000-0003-1010-8157} and linking to 
		     the authors ORCID (see: \url{https://orcid.org}).\\
    {\tt footnote} & Create an author-specific footnote.\\
    {\tt surname}  & Indicate the surname of the author for indexing purposes.\\
    {\tt onclick}  & Provide a URL to visit when clicking on the external link logo
                     \linkicon\ displayed next to the author name:
                     e.g.,~can point to the academic webpage.\\
    {\tt email}    & Define the e-mail address of this author. 
                     Note that at least one e-mail address is required when
                     \texttt{[version=final]} is used.\\
  \end{tabular}
\end{center}

\noindent We \textbf{strongly} recommend that authors enter their
ORCID ID into the paper, because this ensures that they will get
citation credit for their papers. Authors can use the {\tt
  \textbackslash{}surname} macro to indicate what part of the author
name is the surname: this is used for meta-data collection and is
especially useful for cases with middle names \emph{and} double
surnames which might be confusing.

When the URL provided to the {\tt onclick} option contain characters
with a ``special'' meaning in \LaTeX{} they might render incorrectly.
An example with some of the common characters \string~, \%, and \# is
\begin{verbatim}
  onclick = {https://www.webpage.com/\string~\%\#/}
\end{verbatim}
\noindent which displays \authorlink{https://www.webpage.com/\string~\%\#/}
next to the author name.

An example using all the optional arguments is given below. In this case
the author has \verb+inst={1,2}+ to indicate that they are affiliated with
the first and second affiliations that are entered with
\verb+\addaffiliation+:

\begin{verbatim}
\addauthor[orcid    = {0000-0000-0000-0000},
           inst     = {1,2},
           footnote = {Thanks to my supervisor for the support.},
           onclick  = {https://www.mypersonalwebpage.com},
           email    = {alice@accomplished.com},
           surname  = {Accomplished},
          ]{Alice Accomplished}
\end{verbatim}

The author names displayed in the header are constructed automatically
for four authors or less. One can optionally modify this with the
\cmd{authorrunning} macro.  For five or more authors the use of the
\cmd{authorrunning} macro is mandatory. The \cmd{thanks} macro is disabled
inside \cmd{addauthor}, so use the \verb+footnote+ option on \cmd{addauthor}
instead.

\subsection{Affiliations}
Affiliations are listed individually using the \cmd{addaffiliation} command
\emph{after} the last author has been added using \cmd{addauthor}.
There are a number of optional arguments to \cmd{addaffiliation}:

\begin{center}
  \begin{tabular}{l@{\hspace{1cm}}p{0.7\linewidth}}
    {\tt ror}         & Provide the Research Organization Registry (ROR) indentifier
                        for this affiliation (see: \url{https://ror.org}). 
                        This is used for meta-data collection only.\\
    {\tt  department} & Department or suborganization name.\\
    {\tt  street}     & Street address.\\
    {\tt  city}       & City name.\\
    {\tt  state}      & State or province name.\\
    {\tt  postcode}   & Zip or postal code.\\
    {\tt  country}    & Country name. Required for \texttt{[version=final]}.\\
  \end{tabular}
\end{center}
\noindent There is an online tool at
\href{https://publish.iacr.org/funding}{\texttt{publish.iacr.org/funding}}
to help you find ROR identifiers, and you are strongly urged to
include these. Country is required if \verb+[version=final]+ on the
paper.
When provided, only the \verb+city+ and \verb+country+ arguments are used
to display the affliation. The other arguments are used to provide
to indexing agencies.
An example using all the optional arguments would look like:

\begin{verbatim}
\addaffiliation[ror        = {05f950310},
                department = {Computer Security and Industrial Cryptography},
                street     = {Kasteelpark Arenberg 10, box 2452},
                city       = {Leuven},
                state      = {Vlaams-Brabant},
                postcode   = {3001},
                country    = {Belgium}
               ]{KU Leuven}
\end{verbatim}           

\subsection{Funding information}
Authors should use the \texttt{\textbackslash addfunding} macro to
make sure that funding agencies can find papers published under their
sponsorship. An example is:
\begin{verbatim}
\addfunding[fundref = {100000001},
            grantid = {CNS-1237235},
            country = {United States}]{National Science Foundation}
\addfunding[ror     = {00pn5a327},
            country = {United States}]{Rambus}
\end{verbatim}

\noindent In this example, the author acknowledges a grant from the
National Science Foundation and support from Rambus (with no
\texttt{grantid}). The inclusion of funding from an agency without a
\texttt{grantid} might be appropriate if the author simply received
support for a visit.

The complete list of optional arguments for \texttt{\textbackslash addfunding} is:
\begin{center}
  \begin{tabular}{l@{\hspace{1cm}}p{0.7\linewidth}}
    {\tt fundref} & An identifier from the
                    \href{https://publish.iacr.org/funding}{Crossref funder registry}.\\
    {\tt ror}     & An identifier from the 
                    \href{https://publish.iacr.org/funding}{Research Organization Registry} 
                    (ROR). A \texttt{fundref} identifier is preferred 
                    for \texttt{\textbackslash addfunding}.\\
    {\tt country} & The country of the funding agency. \\
    {\tt grantid} & The identifier of the grant that is assigned by the agency 
                    who provided it.
  \end{tabular}
\end{center}
\noindent You can use the online tool at 
\href{https://publish.iacr.org/funding}{\texttt{publish.iacr.org/funding}} to
help you find \texttt{fundref} and \texttt{ror} identifiers.

Note that \cmd{addfunding} \textbf{does not} automatically create footnotes or
an acknowledgements section to identify funding - it only collects the
metadata for indexing. If you wish to include such visible
annotations, you can use the \texttt{footnote} option on
\cmd{addauthor}, or the \cmd{genericfootnote}, or add a separate
acknowledgements section. Some funding agencies have specific
requirements for how they want to be acknowledged in the paper.

\subsection{Footnotes}\label{footnotes}
Authors may be accustomed to using \cmd{thanks} for footnotes
indicating affiliation, email, or funding, but the
\cmd{thanks} macro is disabled and you should use other methods described
in this document.
\begin{itemize}
\item Footnotes on titles are not supported. You should use 
  \cmd{genericfootnote} to place a
  footnote on the first page without a reference. This is useful to
  indicate this is a full / extended version of a published paper, or to
  indicate funding relationships for the authors.  This is an optional
  macro that may be repeated for multiple footnotes.
\item For a footnote on an author, use the \texttt{footnote} option
  on \cmd{addauthor}. This can be used for indicating that the author's
  affiliation for the work was different than their current affiliation,
  or to indicate contact address, or a previous name, etc.
\end{itemize}

Footnotes may be used elsewhere in the paper, but please do not
overuse them.

\subsection{License}
When the \texttt{version=final} document mode is used, the author needs
to provide a supported license.  In all other modes this information
is not required and is ignored if it is provided.  At present the only
acceptable license is \texttt{CC-by}.  An example would look like:

\begin{verbatim}
\license{CC-by}
\end{verbatim}

\subsection{Keywords}
Use \texttt{\textbackslash keywords\{keyword1, keyword2\}} to give a
list of keywords or key phrases. This is an optional macro that should
appear before the abstract.  Individual keywords should be separated
by commas. If the argument to \texttt{\textbackslash keywords}
contains math or macros, then you must supply an additional set of
text-only keywords; for example:

\texttt{\textbackslash keywords[rings, arithmetic on Z]\{rings, arithmetic on \$\textbackslash mathbb\{Z\}\$\}}

\subsection{Abstract}
Abstracts serve several purposes in a journal article, including both
summarization and indexing. An abstract should be a self-contained
mini-document that describes the contributions of the paper. It
should be free of bibliographic references and also free of undefined
terminology introduced in the paper. It is acceptable to use
mathematical notation, but this kind of content is not useful for
indexing.

For this reason, the \texttt{iacrcc} document class uses two kinds
of abstracts. The first (traditional) form of abstract is entered with the
\texttt{abstract} environment as usual.  Note that the keywords should
be given before starting the \texttt{abstract} environment.

For \texttt{final} versions of papers, an additional ``text-only''
abstract is required. This abstract is contained in the
\texttt{textabstract} environment, and should not contain
user-defined macros.
It will be used for indexing and production of
{HTML} pages to describe the paper. As such, it is just as important
as the classical \texttt{abstract} of a paper because it contains a
textual summary that readers will use to decide if the paper is worth
reading. The only difference is that the contents of the
\texttt{textabstract} is constrained on what it may contain.

You may use unicode such as in \verb+Paul Erdős+ or
diacriticals like \verb+F\"ur Elise+.
You may also use inline or display mathematics
in the \texttt{textabstract} environment as well as
(for example) the  \texttt{itemize} environment.
User-defined macros are \emph{not} allowed.
We do not have a complete list of allowed
\LaTeX (which can be succesfully converted to HTML) but
but you will find out when you upload your final version
at \url{https://publish.iacr.org}.

The contents of this environment will be written to a file that ends
with \texttt{.abstract} when you compile your \LaTeX, but will not be
displayed in the final PDF except as metadata. Note that
\verb+\begin{textabstract}+ must appear on a line by itself.

\subsection{Theorems}

The \texttt{iacrcc} class uses the \AmS{} packages to typeset
math.  In particular, it loads the \texttt{amsthm} package, and
predefines the following environments:
\begin{center}
  \ttfamily
\begin{tabular}{l@{\hspace{1cm}}l@{\hspace{1cm}}l}
theorem     & definition & remark \\
proposition & example    & note   \\
problem     & exercise   & case   \\
lemma       & property   &        \\
conjecture  & question   &        \\
corollary   & solution   &        \\
claim       &            &        \\
\end{tabular}
\end{center}

Note that the \texttt{proof} environment automatically adds a QED
symbol at the end of the proof.
If the QED symbol
is typeset at a wrong position, you can force its position with
\verb+\qedhere+.

\section{Auxiliary files}
One goal of the \texttt{iacrcc.cls}
  file is to automate the production of machine-readable metadata in
  separate files. Users of \LaTeX\ will already be used to seeing this
  with the \texttt{.log}, \texttt{.aux}, \texttt{.bbl}, \texttt{.blg},
  \texttt{.toc}, and \texttt{.out} files produced by \BibTeX\ 
  and the \texttt{hyperref} package.  You need not be concerned about
  these, but if your main \LaTeX\ file is called \texttt{main.tex},
  then the extra files that are produced are:
\begin{itemize}
\item a flat text file \texttt{main.meta} containing all metadata
  from the paper.  When you compile
  \texttt{main.tex}, it will produce the metadata from
  \texttt{main.tex}, and when you run \texttt{bibtex} and
  \texttt{latex} again, it will append the citation data from
  \BibTeX into the \texttt{main.meta} file as well.
\item a file \texttt{main.abstract} that contains
  the contents of the abstract for the paper provided with the
  \texttt{textabstract} environment. This will be used to show
  the abstract on the web.
\end{itemize}

\section{Typesetting the Bibliography}
\label{sec:biblio}

Having good bibliographic references is very important for the
visibility of the journal.  Since we don't use a commercial editor,
authors need to make sure themselves that references are standardized
and clean.  We strongly encourage authors to use bibliographic data
from \url{http://www.dblp.org} or \url{https://cryptobib.di.ens.fr/}.
All references should have DOIs if at all possible.

You must use either \BibTeX\ or \BibLaTeX; you may not format your own bibliography.
If you use \BibTeX, then the \texttt{iacrcc} class will load the
\verb+\bibliographystyle{alphaurl}+ style.  You may not change
this. If you use \BibLaTeX, then this is done using
\verb+\documentclass[biblatex]{iacrcc}+ instead of
\verb+\usepackage{biblatex}+; the latter will generate an error
because the \texttt{iacrcc.cls} file loads \BibLaTeX\ with a
specific style.

Here are some example citations: the RSA paper~\cite{RSA78}, the AES
standard~\cite{AES-FIPS}, and \cite{DBLP:conf/crypto/Kocher96}.

For the IACR Communications in Cryptology, you will be required to
upload your \BibTeX\ files rather than just the \texttt{bbl} file.
Many authors use the \texttt{cryptobib} \BibTeX\ files, and you need
not upload those with your paper. They can be referenced as
\texttt{\textbackslash bibliography\{cryptobib/abbrev1,cryptobib/crypto\}}

\section{Package load order}\label{sec:loadorder}

\LaTeX\ suffers from the weakness of having a global namespace for
macros. As a result, it is possible that some packages may overwrite
the definitions of another package that was loaded earlier. The
biggest offender for this seems to be the \texttt{hyperref} package,
which overwrites some basic macros in \LaTeX\ itself. The
\texttt{iacrcc} document class loads \texttt{hyperref}, but it
provides a mechanism for loading packages \emph{after} \texttt{hyperref}. If
the file \texttt{after-hyperref.sty} exists in the directory of your
main file, then it will be included after loading \texttt{hyperref}.
As an example, to load \texttt{cleveref} after \texttt{hyperref}, you
can create a file \texttt{after-hyperref.sty} that contains:
\begin{verbatim}
\RequirePackage{cleveref}
\end{verbatim}
A complete survey of the conflicts between packages is beyond the scope of this document, but
some known conflicts between packages are documented in the
\href{https://github.com/mhelvens/latex-pkgloader/blob/master/pkgloader-recommended.sty}{\texttt{pkgloader}}
package. It is wise to read the documentation for any package you use to make sure
there are no conflicts with other packages loaded by \texttt{iacrcc.cls}.

\section{Some recommendations}\label{sec:options}

\paragraph{\LaTeX{} distribution, and worklow.}  \LaTeX{}
distributions are available on a variety of platforms.  In particular,
we recommend the \href{https://www.tug.org/texlive/}{TeX Live}
distribution, which is updated regularly, includes a large number of
packages, and is available on many platforms. We use the texlive medium
scheme in our cloud service to compile final versions of papers.

\paragraph{Pictures.}
We recommend the use of the \texttt{tikz} package to render pictures.
In particular, a large variety of crypto pictures made with
\texttt{tikz} is available at \href{http://iacr.org/authors/tikz/}{\texttt{iacr.org/authors/tikz/}}

\paragraph{External pictures.}  The \texttt{graphicx} is loaded by the
class, and is recommended for external figures. The submission server
does not support \texttt{svg} format for included graphics, so you
should convert \texttt{svg} files to a supported format.  If possible,
external figures should be in a vector format (PDF or EPS).  Note that
the \verb+\includegraphics+ command will automatically select a file
with what it thinks should be the right extension, so if you write
\verb+\includegraphics{figure}+ and have two files \texttt{figure.gif}
and \texttt{figure.eps}, it will try to select the correct one.

\paragraph{Floats.}
Figure captions should be below the figures, and table captions above
the tables.  The \texttt{float} package loaded by the class should
take care of this automatically.  If want to have several figures side
by side, see the \texttt{[floatrow]} option.

\paragraph{Tables.}
We recommend the \texttt{booktabs} package to typeset tables.

\paragraph{Algorithms.}
We recommend the \texttt{algorithmicx} packages for algorithms (in
particular, \texttt{algpseudocodex} for pseudo-code).

\section{Further information}
If you are a \LaTeX\ novice, you may wish to consult the following documents:
\begin{itemize}
\item General \LaTeX{} documentation, such as the
  \href{http://mirrors.ctan.org/info/lshort/english/lshort.pdf}{(not
    so) short introduction to \LaTeXe};
\item The
  \href{https://mirror.mwt.me/ctan/macros/latex/required/amsmath/amsldoc.pdf}{amsmath
    documentation} is useful for learning how to typeset mathematics.
\end{itemize}

\bibliography{biblio}
\end{document}
