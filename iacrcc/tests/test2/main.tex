% IACR Communications in Cryptology template file
% This file shows how to use the iacrcc class to write a paper.

%% Document mode
% [preprint]      Preprint (no copyright info) -- default mode
% [submission]    Anonymous submission
% [notanonymous]  Keep author names in submission mode
% [final]         Final version
\documentclass[biblatex]{iacrcc}

% When the *final* document mode is used
% the authors need to provide a supported license.
% In all other modes this information is ignored.
% The currently provided ones are { CC-by }
\license{CC-by}

% Include LaTeX packages required by your paper
% \usepackage{}

% Provide the title of the paper
% This should look like:
\title[running  = {The iacrcc class},
       onclick  = {https://github.com/IACR/latex},
       subtitle = {A Template}
      ]{How to Use the IACR Communications in Cryptology Cl\r ass}

% Where the options in square brackets “[ ]”are optional and control the following:
% running: the running title displayed in the headers
% onclick: define what to do when clicking on the title. 
%          This could be used to point to a webpage associated with this paper. 
% subtitle: provide a subtitle

% Define authors and affiliations
% Authors are listed individually using the \addauthor tag followed by a list of affiliations.
% The idea is that every author makes a separate call to this command.
% This should look like:
% \addauthor[inst      = {1,2},
%            orcid     = 0000-0000-0000-0000,
%            footnote  = {Thanks to my supervisor for the support.},
%            onclick   = {https://www.mypersonalwebpage.com}
%           ]{Alice Accomplished}
% Where the options in square brackets “[ ]”are optional 
% and control the following:
% inst:     a numerical list pointing to the index of the institutaion 
%           in the affiliation array.
% orcid:    create a small clickable orcid logo next to the authors name 
%           linking to the authors ORCID iD see: orcid.org.
% footnote: create an author-specific footnote.
% onclick:  define what to do when clicking on the author name: 
%           e.g., can point to the academic webpage.
% email:    define the e-mail address of this author.
\addauthor[orcid    = {0000-0003-1010-8157},
           inst     = {1},
           onclick  = {https://www.joppebos.com},
	   footnote = {This is an example fö{\"o}tnote.},
           email    = {joppe.bos@nxp.com},
	   surname  = {B{\"o}s}
	  ]{Joppe W. B{\"o}s}

\addauthor[orcid   = {0000-0001-7890-5430},
           inst    = {2},
           email   = {test2@digicrime.com},
	   surname = {McCurley}
          ]{Kevin S. McCurley}

% Affiliations are listed individually using the \affiliations command 
% *after* the (list of) authors using \addauthor
% This should look like (full example):
% \affiliation[ror        = 05f950310,
%              onclick    = {http://www.kuleuven.be/english},
%              department = {Computer Security and Industrial Cryptography},              
%              street     = {Kasteelpark Arenberg 10, box 2452},
%              city       = {Leuven},
%              state      = {Vlaams-Brabant},
%              postcode   = {3001},
%              country    = {Belgium}
%             ]{KU Leuven}
% Where the options in square brackets “[ ]”are optional and control 
% the following (optional information is mainly used for meta-data collection):
% ror:        provide the Research Organization Registry (ROR) indetifier 
%             for this affiliation (see: ror.org). This is used for meta-data 
%             collection only.
% onclick:    define what to do when clicking on the affiliation name: 
%             e.g., can point to the affiliation webpage.
% department: department or suborganization name
% street:     street address
% city:       city name
% state:      state or province name
% postcode:   zip or postal code
% country:    country name

\affiliation[ror     = 031v4g827,
             onclick = {https://www.nxp.com},
             street  = {Interleuvenlaan 80},
             city    = {Leuven},
             postcode= {3001},
             country = {Belgium}
	    ]{NXP S\v{e}mïc{\"o}ndúct\o{}rs}
\affiliation[country={Elbonia}]{Self}


% A footnote can be placed on the front page without a symbol / numbering using:
\genericfootnote{This is the f\d{u}ll version of our paper published at \c{c}rypto.}

\addbibresource{test2.bib}
\begin{document}

\maketitle

% Provide the keywords *before* the abstract
\keywords{T\'emplate, LaTeX, IACR}

% Provide the abstract of your paper
\begin{abstract}
This \LaTeX{} template will show you how to use the IACR
Communications in Cryptology Class style file.  Except for the
metadata collection and options related to authors, title and
affiliation this class file is identical to the IACR Transactions
class file (the iacrtrans class) written by Ga{\"e}tan Leurent.

NOTE: Your abstract may contain inline mathematics: \(\alpha=b\) and $b=\delta$
as well as displayed mathematics:
\[
  \alpha = \delta
  \]
but please avoid using macros that you define yourself in the abstract.
\end{abstract}

% The content of the paper starts here
\section{Introduction}
This is the template showing how to use the IACR Communications in Cryptology \LaTeX{} class. 

\section{Title Page}
To place a footnote on the first page without a reference (e.g., to indicate this is a full / extended version of a published paper) 
you can use the {\tt \textbackslash{}genericfootnote} command. 

\subsubsection*{The AddAuthor Command}
Authors are listed individually using the {\tt \textbackslash{}addauthor} command. 
There are a number \emph{optional} options to {\tt \textbackslash{}addauthor}:

\begin{tabular}{l@{\hspace{1cm}}p{0.7\linewidth}}
{\tt inst} & A numerical list pointing to the index of the institutaion in the affiliation array.\\
{\tt orcid} & Create a small clickable orcid logo next to the authors name looking like \orcidlink{0000-0003-1010-8157} and linking to the authors ORCID iD (see: \url{https://orcid.org}.\\
{\tt footnote} & Create an author-specific footnote.\\
{\tt onclick} & Define what to do when clicking on the author name: e.g.,~can point to the academic webpage.\\
{\tt email} & Define the e-mail address of this author.\\
\end{tabular}

\noindent Moreover, one can utilize the {\tt \textbackslash{}surname} macro to indicate what part of the author name is the surname:
this is used for meta-data collection and is especially useful for cases with middle names \emph{and} double 
surnames which might be confusing. 

An example using all the optional options would look like:

\begin{verbatim}
\addauthor[orcid    = 0000-0000-0000-0000,
           inst     = {1,2},
           footnote = {Thanks to my supervisor for the support.},
           onclick  = {https://www.mypersonalwebpage.com},
           email    = {alice@accomplsihed.com},
       ]{Alice \surname{Accomplished}}
\end{verbatim}

\subsubsection*{The Affiliation Command}
Affiliations are listed individually using the {\tt \textbackslash{}affiliations} command \emph{after}
the (list of) authors using {\tt \textbackslash{}addauthor}.
There are a number \emph{optional} options to {\tt \textbackslash{}affiliation}:

\begin{tabular}{l@{\hspace{1cm}}p{0.7\linewidth}}
{\tt ror} & Provide the Research Organization Registry (ROR) indetifier for this affiliation (see: \url{https://ror.org}). This is used for meta-data collection only.\\
{\tt onclick} & Define what to do when clicking on the affiliation name: e.g.,~can point to the affiliation webpage.\\
{\tt  department} & Department or suborganization name.\\
{\tt  street} & Street address.\\
{\tt  city} & City name.\\
{\tt  state} & State or province name.\\
{\tt  postcode} & zip or postal code.\\
{\tt  country} & Country name.\\
\end{tabular}

An example using all the optional options would look like:

\begin{verbatim}
\affiliation[ror        = 05f950310,
             onclick    = {http://www.kuleuven.be/english},
             department = {Computer Security and Industrial Cryptography},              
             street     = {Kasteelpark Arenberg 10, box 2452},
             city       = {Leuven},
             state      = {Vlaams-Brabant},
             postcode   = {3001},
             country    = {Belgium}
            ]{KU Leuven}
\end{verbatim}
            
\subsubsection*{The Title Command}
The title is provided using {\tt \textbackslash{}title}.
There are a number \emph{optional} options to {\tt \textbackslash{}title}:

\begin{tabular}{l@{\hspace{1cm}}p{0.7\linewidth}}
{\tt running} & The running title displayed in the headers.\\
{\tt onclick} & Define what to do when clicking on the title. This could be used to point to a webpage associated with this paper. \\
{\tt subtitle} & Provide a subtitle.\\
\end{tabular}

An example using all the optional options would look like:

\begin{verbatim}
\title[running  = {The iacrcc class},
       onclick  = {https://github.com/IACR/latex},
       subtitle = {A Template}
      ]{How to Use the IACR Communications in Cryptology Class}
\end{verbatim}

\subsubsection*{The License Command}
When the ``final'' document mode is used the author needs to provide a supported license.
In all other modes this information is ignored.
The currently provided set of licences are \mbox{\{ CC-by \}}.
An example would look like:

\begin{verbatim}
\license{CC-by}
\end{verbatim}

\section{Bibliography}
Citing papers is done in the usual way using BibTeX or \texttt{biblatex}
commands. For example: the RSA paper~\cite{RSA78}, \cite{sample},
\cite{DBLP:journals/joc/ChillottiGGI20}, \cite{latexproject,fancynames} and \cite{DBLP:conf/crypto/Kocher96}.

It is highly encouraged to use CyptoBib from \url{https://cryptobib.di.ens.fr}

\printbibliography
% NOTES
% - Download abbrev3.bib and crypto.bib from https://cryptobib.di.ens.fr/
% - Use bilbio.bib for additional references not in the cryptobib database.
%   If possible, take them from DBLP.

\end{document}
