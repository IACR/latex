%%%% IACR Transactions TEMPLATE %%%%
% This file shows how to use the iacrtrans class to write a paper.
% Written by Gaetan Leurent gaetan.leurent@inria.fr (2020)
% Public Domain (CC0)


%%%% 1. DOCUMENTCLASS %%%%
\documentclass[version=final]{iacrcc}
%%%% NOTES:
% - Add "spthm" for LNCS-like theorems

\errorcontextlines=5
\newcommand\niceguy{Fester Bestertester}
%%%% 2. PACKAGES %%%%
\usepackage{lipsum} % Example package -- can be removed
\usepackage{amsmath}
\usepackage[TU]{fontenc}
\usepackage[T1]{fontenc}
%%%% 3. AUTHOR, INSTITUTE %%%%

\license{CC-by}

% NOTE: There are two affiliations, referred to by 1,2
\addauthor[orcid=0000-0003-1010-8157,inst={1,2},footnote={Work done while working for XYZ.}]{Alice Accomplished\thanks{Work done while working at ABC.}}
% NOTE: Only one affiliation for this author.
\addauthor[inst={1},footnote={Thanks to my mom!},
  email={bad@example.com}]{Bob Badenuff}
\addauthor[inst={3,2,1}]{Tancr{\`e}de Lepoint}
\addaffiliation[ror=02t274463,
  country={United States}]{University of California, Santa Barbara}
\addaffiliation[country={Elbonia}]{University of Antartica}
\addaffiliation[country={Turkey}]{Bo{\u g}azi{\c c}i University}

\genericfootnote{This is the full-version of our previous work.}

%%%% 4. TITLE %%%%
% Note that \thanks and \footnote are forbidden in \title. Please use
% \genericfootnote.
\title[running={Thoughts on binary functions},
  plaintext={Thoughts about "binary" functions and \$\ on $GF(p)$ by Fester Bestertester at 30°C}]{%
  Thoughts about "binary" functions and \$\ on $GF(p)$ by \niceguy\ at 30°C}
\begin{document}

\maketitle

%%%% 5. ABSTRACT %%%%
\begin{abstract}
  In this paper we prove that the One-Time-Pad has perfect security, unless you use
  double encryption with {$\mathsf{SingleKey}$}.

  \lipsum[8]
\end{abstract}


%%%% 6. PAPER CONTENT %%%%
\section{Introduction}

Widely used primitives like the AES~\cite{AES} do not have perfect
security, and can be analysed with linear
cryptanalysis~\cite{EC:Matsui93}, differential
cryptanalysis~\cite{JC:BihSha91}, or differential power
analysis~\cite{C:KocJafJun99}.  We show that the One-Time-Pad is
unconditionally secure in \autoref{sec:main}. Let's not forget
the work by John~\cite{vonNeumann}.

Note that we can tolerate ampersands~\cite{Dalheimer02} and \TeX
character codes~\cite{Bohme10} and math in
title~\cite{ACISP:MurPla19,ACISP:LYLF19,ACISP:WeiSteSha03,CCS:BHKNRS19,ACNS:DurHugVau20}.

\lipsum[9]

\section{Main Result}\label{sec:main}

\lipsum

\section{Section heading with math in it $a=b$}
\lipsum

This\footnote{is a footnote}.

This is the unicode degree symbol: 30°
%%%% 8. BILBIOGRAPHY %%%%
\bibliography{test1}
%%%% NOTES
% - Download abbrev3.bib and crypto.bib from https://cryptobib.di.ens.fr/
% - Use biblio.bib for additional references not in the cryptobib database.
%   If possible, take them from DBLP.

\end{document}
