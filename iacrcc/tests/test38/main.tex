%%%% 1. DOCUMENTCLASS %%%%
\documentclass[version=final]{iacrcc}
%%%% NOTES:
\usepackage{listings}
\usepackage{multicol}
\newcommand\niceguy{Fester Bestertester}
\license{CC-by}

% NOTE: There are two affiliations, referred to by 1,2
\addauthor[orcid={0000-0003-1010-8157},
           email={Alice@example.com},
           inst={1},
           footnote={Work done while working for XYZ.}]{Alice Accomplished}
\addaffiliation[ror=02t274463,country={United States}]{University of California, Santa Barbara}

\title[running={Thoughts on binary functions},
  subtitle={A story about Fester is here},
  plaintext={Thoughts about "binary" functions and \$\ on $GF(p)$ by Fester Bestertester at 30°C}]{%
  Thoughts about "binary" functions and \$\ on $GF(p)$ by \niceguy\ at 30°C}
\begin{document}

\maketitle

%%%% 5. ABSTRACT %%%%
\begin{abstract}
  In this test we make sure that it will compile with line numbers in
  the presence of the listings package.
\end{abstract}
\begin{textabstract}
  In this test we make sure that it will compile with line numbers in
  the presence of the listings package.
\end{textabstract}
\section{Just a section}
With not much in it.

This was an example from overleaf:
\begin{lstlisting}
import numpy as np
    
def incmatrix(genl1,genl2):
    m = len(genl1)
    n = len(genl2)
    M = None #to become the incidence matrix
    VT = np.zeros((n*m,1), int)  #dummy variable
    
    #compute the bitwise xor matrix
    M1 = bitxormatrix(genl1)
    M2 = np.triu(bitxormatrix(genl2),1) 

    for i in range(m-1):
        for j in range(i+1, m):
            [r,c] = np.where(M2 == M1[i,j])
            for k in range(len(r)):
                VT[(i)*n + r[k]] = 1;
                VT[(i)*n + c[k]] = 1;
                VT[(j)*n + r[k]] = 1;
                VT[(j)*n + c[k]] = 1;
                
                if M is None:
                    M = np.copy(VT)
                else:
                    M = np.concatenate((M, VT), 1)
                
                VT = np.zeros((n*m,1), int)
    
    return M
\end{lstlisting}
This is a line after the listing.
\begin{multicols}{4}
\lstinputlisting{XORseq_RS}
\end{multicols}
This will resume having line numbers since it finished a block in \texttt{lstinputlisting}.
Yet another line in the file to create yet another line number.
\end{document}	
