%%%% IACR Transactions TEMPLATE %%%%
% This file shows how to use the iacrtrans class to write a paper.
% Written by Gaetan Leurent gaetan.leurent@inria.fr (2020)
% Public Domain (CC0)


%%%% 1. DOCUMENTCLASS %%%%
\documentclass[final]{iacrcc}
%%%% NOTES:
% - Add "spthm" for LNCS-like theorems

\errorcontextlines=5

%%%% 2. PACKAGES %%%%
\usepackage{lipsum} % Example package -- can be removed

%%%% 3. AUTHOR, INSTITUTE %%%%

\addauthor[orcid=0000-0003-1010-8157,inst={1},onclick={https://www.joppebos.com}]{Joppe W. \surname{Bos}}
\addauthor[inst={1,2},onclick={www.test.com}]{Kevin S. \surname{McCurley}}
\affiliation[ror=02t274463,onclick={https://www.nxp.com}]{NXP Semiconductors\email{joppe.bos@nxp.com}}
\affiliation{Self}


%%%% 4. TITLE %%%%
% Note that \thanks and \footnote are forbidden in \title. Please use
% \genericfootnote.
%\title[IACR Communications in Cryptology Class]{How to Use the IACR Communications in Cryptology \LaTeX Class}

\title[IACR Communications in Cryptology Class]{How to Use the IACR Communications in Cryptology Class}

\begin{document}

\maketitle


%%%% 5. ABSTRACT %%%%
\begin{abstract}
  In this paper we prove that the One-Time-Pad has perfect security.

  \lipsum[8]
\end{abstract}


%%%% 6. PAPER CONTENT %%%%
\section{Introduction}
This is the template showing how to use the IACR Communications in Cryptology \LaTeX 
class. 

\section{Title Page}
To place a footnote on the first page without a reference (e.g., to indicate this is a full / extended version of a published paper) 
you can use the {\tt \textbackslash{}genericfootnote} command. 

\subsubsection*{The Author Command}
Authors are listed individually using the {\tt \textbackslash{}author} command. 
There are a number \emph{optional} options to {\tt \textbackslash{}author}:

\begin{tabular}{l@{\hspace{1cm}}p{0.7\linewidth}}
{\tt inst} & A numerical list pointing to the index of the institutaion in the affiliation array.\\
{\tt orcid} & Create a small clickable orcid logo next to the authors name looking like \orcidlink{0000-0003-1010-8157} and linking to the authors ORCID iD (see: \url{https://orcid.org}.\\
{\tt footnote} & Create an author-specific footnote.\\
{\tt onclick} & Define what to do when clicking on the author name: e.g.,~can point to the academic webpage.\\
\end{tabular}

\noindent Moreover, one can utilize the {\tt \textbackslash{}surname} macro to indicate what part of the author name is the surname:
this is used for meta-data collection and is especially useful for cases with middle names \emph{and} double 
surnames which might be confusing. 

An example using all the optional options would look like:

\begin{verbatim}
\author[orcid=0000-0003-1010-8157,
        inst={1,2},
        footnote={Thanks to my supervisor for the support.},
        onclick={https://www.mypersonalwebpage.com}
       ]{Alice \surname{Accomplished}}
\end{verbatim}

\subsubsection*{The Affiliation Command}
Affiliations are listed individually using the {\tt \textbackslash{}affiliations} command \emph{after}
the (list of) authors using {\tt \textbackslash{}author}.

\begin{tabular}{l@{\hspace{1cm}}p{0.7\linewidth}}
{\tt ror} & Provide the Research Organization Registry (ROR) indetifier for this affiliation (see: \url{https://ror.org}). This is used for meta-data collection only.\\
{\tt onclick} & Define what to do when clicking on the affiliation name: e.g.,~can point to the affiliation webpage.\\
\end{tabular}

An example using all the optional options would look like:

\begin{verbatim}
\affiliation[ror=05f950310,
             onclick={http://www.kuleuven.be/english}
            ]{KU Leuven}
\end{verbatim}

\section{Acknowledgements}
Except for the metadata collection and options related to authors, title and affiliation this 
class file is identical to the IACR Transactions class file (the iacrtrans class) written by Ga{\"e}tan Leurent.

%%%% 8. BILBIOGRAPHY %%%%
%\bibliographystyle{iacrcc}
%\bibliography{cryptobib/abbrev3,cryptobib/crypto}
%%%% NOTES
% - Download abbrev3.bib and crypto.bib from https://cryptobib.di.ens.fr/
% - Use bilbio.bib for additional references not in the cryptobib database.
%   If possible, take them from DBLP.

\end{document}
