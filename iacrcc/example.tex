%%%% IACR Transactions TEMPLATE %%%%
% This file shows how to use the iacrtrans class to write a paper.
% Written by Gaetan Leurent gaetan.leurent@inria.fr (2020)
% Public Domain (CC0)


%%%% 1. DOCUMENTCLASS %%%%
\documentclass[final]{iacrcc}
%%%% NOTES:
% - Add "spthm" for LNCS-like theorems

\errorcontextlines=5

%%%% 2. PACKAGES %%%%
\usepackage{lipsum} % Example package -- can be removed


%%%% 3. AUTHOR, INSTITUTE %%%%

\iftrue

\author[orcid=0000-0003-1010-8157,inst={1,2}]{Alice Accomplished}
%\author[inst={1},footnote={Thanks to my mom!}]{Bob Badenuff}
%\author[orcid=0000-0002-2447-4329]{Charlie C}
\affiliation[ror=02t274463]{University of California, Santa Barbara}
\affiliation{University of Second Choice}

\genericfootnote{This is the full-version of our previous work.}

\else
% This is the format we want to have:
\author{Alice Accomplished
  \affiliation{1,2}}
\author{Bob Badenuff
  \affiliation{1}
  \sponsor{1}}
\affiliation{University of California, Santa Barbara\department{Department of Computer Science}\RORID{02t274463}}
\affiliation{University of Second Choice}

%  Authors are listed individually in the \author tag, with a list of affiliations.
\author{Alice Accomplished\inst{1,2}\sponsor{1}\email{alice@usc.edu}} (two affiliations)
\author{Bob Badenuff\inst{2}} (a single affiliation)
%  Affiliations are listed as an array. One \affiliation per institution
\affiliation{University of California, Santa Barbara\department{Department of Computer Science}\RORID{02t274463}}
\affiliation{University of Second Choice}
% Sponsors are listed individually, and are referenced from the author by \sponsor
\sponsor{National Security Agency\grantid{666-2021}}

\fi

%%%% 4. TITLE %%%%
\title[Thoughts on binary functions]{Thoughts about binary functions on $GF(p)$}

\begin{document}

\maketitle


%%%% 5. ABSTRACT %%%%
\begin{abstract}
  In this paper we prove that the One-Time-Pad has perfect security.

  \lipsum[8]
\end{abstract}


%%%% 6. PAPER CONTENT %%%%
\section{Introduction}

Widely used primitives like the AES~\cite{AES} do not have perfect
security, and can be analysed with linear
cryptanalysis~\cite{EC:Matsui93}, differential
cryptanalysis~\cite{JC:BihSha91}, or differential power
analysis~\cite{C:KocJafJun99}.  We show that the One-Time-Pad is
unconditionally secure in \autoref{sec:main}.

\lipsum[9]

\section{Main Result}\label{sec:main}

\lipsum

This\footnote{is a footnote}.

%%%% 8. BILBIOGRAPHY %%%%
\bibliographystyle{alpha}
\bibliography{cryptobib/abbrev3,cryptobib/crypto,biblio}
%%%% NOTES
% - Download abbrev3.bib and crypto.bib from https://cryptobib.di.ens.fr/
% - Use bilbio.bib for additional references not in the cryptobib database.
%   If possible, take them from DBLP.

\end{document}
